\documentclass{amsart}
\usepackage{amsmath,amsfonts,amssymb,latexsym}
\usepackage[margin=1in]{geometry}

\newtheorem{theorem}{Theorem}[section]
\newtheorem{corollary}[theorem]{Corollary}
\newtheorem{definition}[theorem]{Definition}
\newtheorem{lemma}[theorem]{Lemma}
\newtheorem{question}[theorem]{Question}
\newtheorem{proposition}[theorem]{Proposition}
\newtheorem{remark}[theorem]{Remark}
\newtheorem{example}[theorem]{Example}


\newcommand{\bfp}{\mathbf{p}}
\newcommand{\bfs}{\mathbf{s}}
\newcommand{\bfx}{\mathbf{x}}

\newcommand{\cA}{\mathcal{A}}
\newcommand{\cF}{\mathcal{F}}
\newcommand{\cG}{\mathcal{G}}

\newcommand{\CC}{\mathbb{C}}
\newcommand{\FF}{\mathbb{F}}
\newcommand{\kk}{\Bbbk}
\newcommand{\QQ}{\mathbb{Q}}
\newcommand{\RR}{\mathbb{R}}
\newcommand{\ZZ}{\mathbb{Z}}

\newcommand{\diag}{\operatorname{diag}}
\newcommand{\Hom}{\operatorname{Hom}}
\renewcommand{\max}{\operatorname{max}}
\newcommand{\Spec}{\operatorname{Spec}}

\newcommand{\rra}{\rightrightarrows}

\newcommand{\erase[1]}{{}}

\title{Symplectic Groupoids for Cluster Manifolds}

\begin{document}
\begin{abstract}
  We construct a symplectic groupoid integrating a log-canonical Poisson structure on a cluster manifold.
\end{abstract}
\maketitle
Outline
\begin{enumerate}
	\item Intro to Poisson manifolds, symplectic groupoid and Poisson spray
	\item Intro to cluster algebra and compatible Poisson structures 
	\item Cluster symplectic groupoid
	\item Totally positive cluster manifolds (definition of manifolds with corners [check D Joyce], associahedron of type A and generalized associahedron)
	\item Symplectic topology of the groupoid, and examples
\end{enumerate}

%%%%%%%%%%%%%%%%%%%%%%
\section{Introduction}


%%%%%%%%%%%%%%%%%%%%%%%%%%
\section{Poisson Geometry}
In this section we recall the various definitions of Poisson manifolds and the construction of symplectic groupoids from the Poisson spray \cite{MR2900786}.

\begin{definition}
A smooth Poisson manifold is a smooth manifold $M$ equipped with one of the following three equivalent structures:
  \begin{enumerate}
    \item A Lie bracket (called a Poisson bracket)
      $$
				\{\cdot, \cdot\}: C^\infty(M) \times C^\infty(M) \to C^\infty(M)
			$$
				satisfying the Leibniz rule
			$$
				\{fg, h\} = f\{g,h\} + g\{f,h\}.
			$$
		\item A Poisson bivector $\pi \in \mathfrak{X}^2(M)$ such that $[\pi, \pi] = 0$ where
			$$
				[\cdot, \cdot]: \mathfrak{X}^p(M) \times \mathfrak{X}^q(M) \to \mathfrak{X}^{p+q-1}(M)
			$$
			is the Schouten-Nijenhuis bracket.
		\item A Poisson differental operator $d_\pi: \mathfrak{X}^p(M) \to \mathfrak{X}^{p+1}(M)$ with $d_\pi^2 = 0$.
\end{enumerate}	
	A holomorphic Poisson manifold is analogous where $M$ is a complex manifold.
\end{definition}

\begin{remark}
	We shall denote a Poisson manifold by either $(M, \pi)$ or $(M, \{,\})$.
	\begin{enumerate}
		\item The Poisson bivector is related to the Poisson bracket by the formula: $\{f, g\} = \pi (df \otimes dg)$, for $f, g\in C^\infty(M)$.
		\item The Poisson differential operator is related to the Poisson bivector by the formula: $d_\pi X = [\pi, X]$ for $X \in \mathfrak{X}^p(M)$.
	\end{enumerate}
\end{remark}

The notion of the symplectic groupoid of a Poisson manifold was introduced independently by Weinstein \cite{MR866024}, Karas\"{e}v \cite{MR1008479} and Zakrzewski \cite{MR1081010, MR1081011}. It is closely related to Poisson sigma models \cite{MR1938552} and quantization \cite{MR2417440}. As a Lie groupoid, the symplectic groupoid integrates the Poisson Lie algebroid $\pi^\sharp: T^*M \to TM$ \cite{MR866024}.

\begin{definition}
A groupoid $G \rightrightarrows M$ consists of two sets $G$ and $M$ with the following maps:
	\begin{enumerate}
		\item a surjective source map $\alpha: G \to M$ and a surjective target map $\beta: G \to M$;
		\item an injective identity map $\mathtt{1}: M \to G, \enskip x \mapsto \mathtt{1}_x$;
		\item an associative multiplication map $m: G {_\beta \times_\alpha} G \to G, \enskip (g, h) \mapsto gh$;
		\item and an involutive inversion map $i: G \to G, \enskip g \mapsto g^{-1}$
	\end{enumerate}
that satisfies the following properties:
	\begin{enumerate}
		\item $\alpha(\mathtt{1}_x) = \beta(\mathtt{1}_x) = x$;
		\item $\alpha(gh) = \alpha(g), \enskip \beta(gh) = \beta(h)$;
		\item $\alpha(g^{-1}) = \beta(g), \enskip \beta(g^{-1}) = \alpha(g)$;
		\item $(\mathtt{1}_x)^{-1} = \mathtt{1}_x$.
	\end{enumerate}
A Lie groupoid $G \rightrightarrows M$ has the following additional properties:
	\begin{enumerate}
		\item $G$ and $M$ are smooth manifolds;
		\item the source $\alpha: G \to M$ and the target $\beta: G \to M$ are surjective submersions;
		\item the multiplication map $m: G {_\beta \times_\alpha} G \to G$ is smooth;
		\item the inversion map $i: G \to G$ is smooth.
	\end{enumerate}
A holomorphic groupoid $G \rightrightarrows M$ is analogous where $G$ and $M$ is a complex manifold and the struture maps are holomorphic.
\end{definition}

\begin{definition}
For a Poisson manifold $(M, \pi)$, a symplectic groupoid is a symplectic manifold $(G, \omega)$ with a Lie groupoid structure $G \rightrightarrows M$ such that
	\begin{enumerate}
		\item the source $\alpha: (G, \omega) \to (M, \pi)$ and the target $\beta: (G, \omega) \to (M, \pi)$ are Poisson maps;
		\item the graph of the multiplication map $\Gamma_m = \{(g, h, gh) \in G \times G \times G\}$ is a Lagrangian submanifold of $(G \times G \times G, \omega \oplus \omega \oplus -\omega)$.
	\end{enumerate}
\end{definition}

The conditions for the existence of symplectic groupoids integrating a given Poisson manifold, and more generally the existence of Lie groupoids integrating a given Lie algebroid, were found by Crainic and Fernandes \cite{MR1973056, MR2128714}. There are a few notable examples of symplectic groupoids, e.g. the symplectic double groupoid of Poisson Lie groups \cite{MR1054741} and the blow-up construction of log symplectic manifolds \cite{MR3214314}, but in general it is difficult to find examples of symplectic groupoids. There is, however, a local construction of symplectic groupoids by Crainic-Marcut \cite{MR2900786} and Cabrera-Marcut-Salazar \cite{CMS17}, which utilizes the notion of a Poisson spray.

\begin{definition}
For a Poisson manifold $(M, \pi)$, a Poisson spray is a vector field $X \in \mathfrak{X}(T^*M)$ is a Poisson spray if
	\begin{enumerate}
		\item for $(x,p) \in T^*M$,
			$$
				(\tau_M)_*\left(X_{(x,p)} \right) = \pi^\flat(p)
			$$
			where $(\tau_M): T^*M \to M$ is the projection;
		\item
			$X$ is homogeneous of degree 1, i.e.
			$$
				(m_\lambda)_*(X) = \lambda X 
			$$
			where $m_\lambda: T^*M \to T^*M, \enskip (x,p) \mapsto (x,\lambda p)$ is the fiberwise scaling map.
	\end{enumerate}
\end{definition}

\begin{theorem} \cite{MR2900786, CMS17} \label{thm:poissp}
For a Poisson manifold $(M, \pi)$ with a Poisson spray $X \in \mathfrak{X}(T^*M)$, a neighbourhood $U$ of the zero section of $T^*M$ is a local symplectic groupoid over $(M, \pi)$ with the following structures:
	\begin{enumerate}
		\item the source map $\alpha = \tau_M: U \to M$ is the bundle projection ;
		\item the target map is
			$$
				\beta: U \to M, \qquad \beta = \tau_M \circ \varphi_X^1
			$$
		where $\varphi_X^1: T^*M \to T^*M$ is the time-$1$-flow of the Poisson spray $X$;
		\item the multiplication is the concatenation of the flow of $X$; and
		\item the symplectic form on $U$ is
			$$
				\overline{\omega} = \int_{0}^{1} (\varphi_X^s)^*\omega_0 ds.
			$$
	\end{enumerate}
\end{theorem}

\begin{remark}
	By a local symplectic groupoid $G \rightrightarrows M$, we mean that the multiplication $m: G {_\beta \times_\alpha} G \to G$ may not be defined on the entirety of its domain. In this particular case, the time-$1$-flow of the Poisson spray $X$ may be outside the neighbourhood $U$ of the zero section.
	
	In general, the local symplectic groupoid structure cannot be extended to the total space of $T^*M$. First of all, the Poisson spray $X$ may not be complete; secondly, the flow of the Poisson spray $X$ may loops; and finally, the 2-form $\overline{\omega}$, though non-degenerate near the zero section of $T^*M$, may not be non-degenerate on the total space of $T^*M$.
\end{remark}

\begin{corollary}
	Given a Poisson spray $X \in \mathfrak{X}(T^*M)$ on a Poisson manifold $(M, \pi)$, there is actually a 1-parameter family of local symplectic groupoid structure on a neighourhood $U$ of the zero section of $T^*M$ with the following structures:
	\begin{enumerate}
		\item the source map $\alpha_t = \tau_M: U \to M$ is the bundle projection;
		\item the target map is
			$$
				\beta: U \to M, \qquad \beta_t = \tau_M \circ \varphi_X^t
			$$
		where $\varphi_X^t: T^*M \to T^*M$ is the time-$t$-flow of the Poisson spray $X$;
		\item the multiplication is the concatenation of the time-$t$-flow of $X$; and
		\item the symplectic form on $U$ is
			$$
				\overline{\omega}_t = \frac{1}{t}\int_{0}^{t} (\varphi_X^s)^*\omega_0 ds.
			$$
	\end{enumerate}
The local symplectic groupoid $(U, \overline{\omega}_t)$ integrates the Poisson manifold $(M, t\pi)$ for $0 \leq t\leq 1$. In particular, the local symplectic groupoid in Theorem~\ref{thm:poissp} is the case $t = 1$.
\end{corollary}

\begin{remark} \label{remark:PoiSpLoc}
In local coordinates $T^*\mathbb{R}^n = \{(x_1, \ldots, x_n, p_1, \ldots, p_n)\}$, if the Poisson structure
$$
	\pi = \sum_{i > j} \pi_{ij} \frac{\partial}{\partial x_i} \wedge \frac{\partial}{\partial x_j},
$$
then a Poisson spray is of the form
$$
	X = \sum_{i > j} \pi_{ij} p_i \frac{\partial}{\partial x_j} - \sum_{i > j} \pi_{ij} p_j \frac{\partial}{\partial x_i} + \sum_i f_i \frac{\partial}{\partial p_i}
$$
where $f_i$'s are fiberwise quadratic functions on $T^*\mathbb{R}^n$.
\end{remark}

\begin{example} \label{ex:PoiSpLogC}
Let $\cF$ be either $\RR$ or $\CC$, and let $\Omega_{ij}$ be a skew-symmetric matrix over $\cF$. For the 1-parameter family of Poisson structures on $\cF^n$:
$$
	\pi_t = t\sum_{i > j} \Omega_{ij} x_i x_j\frac{\partial}{\partial x_i} \wedge \frac{\partial}{\partial x_j}, \qquad 0\leq t \leq 1,
$$
there is a 1-parameter family of symplectic groupoid $G_t \rra \cF^n$, for $0 \leq t \leq 1$ with the following structures:
	\begin{enumerate}
		\item $G_t = T^*\cF^n$;
		\item the source map is the bundle projection
			$$
				\alpha_t: T^*\cF^n \to \cF^n, \qquad (p_1, \ldots, p_n, x_1, \ldots, x_n) \mapsto (x_1, \ldots, x_n);
			$$
		\item the target map
			$$
				\begin{aligned}
				\beta_t: & T^*\cF^n \to \cF^n, \\
				& (p_1, \ldots, p_n, x_1, \ldots, x_n) \mapsto \left(\exp\left(t\sum\limits_{1 < i \leq n} \Omega_{i1} x_ip_i \right)x_1, \ldots, \exp\left(t\sum\limits_{1 < i \leq n} \Omega_{in} x_ip_i \right)x_n\right);
				\end{aligned}
			$$
		\item the multiplication map
			$$
				\begin{aligned}
				m_t: & T^*\cF^n {_{\beta_t}\times_{\alpha_t}} T^*\cF^n \to T^*\cF^n, \\
				& \left(\left(p_1, \ldots, p_n, x_1, \ldots, x_n\right), \left(p'_1, \ldots, p'_n, \exp\left(t\sum\limits_{1 < i \leq n} \Omega_{i1} x_ip_i \right)x_1, \ldots, \exp\left(t\sum\limits_{1 < i \leq n} \Omega_{in} x_ip_i \right)x_n\right)\right) \\
				& \mapsto \left(\exp\left(t \sum\limits_{1\leq i\leq n}\Omega_{i1}x_i p_i\right) p'_1 + p_1, \ldots, \exp\left(t \sum\limits_{1\leq i\leq n}\Omega_{in}x_i p_i\right) p'_n + p_n, x_1, \ldots, x_n \right);
				\end{aligned}
			$$
		\item and the symplectic form $\overline{\omega}_t$
		$$
			\sum_{1\leq i \leq n} dp_i \wedge dx_i
			- t \left(
			\sum_{1 \leq i, j \leq n} \Omega_{ij}x_i p_j d p_i \wedge d x_j 
			+ \sum_{1 \leq j \leq n, j < i \leq n} \Omega_{ij}p_ip_j d x_i \wedge d x_j
			+ \sum_{1 \leq j \leq n, j < i \leq n} \Omega_{ij}x_ix_j d p_i \wedge d p_j
			\right).
		$$
	\end{enumerate}
	In the case of $\cF = \RR$, this symplectic groupoid is realized by choosing the Poisson spray
	$$
		X = \sum_{1 \leq j \leq n, 1 < i \leq n}\Omega_{ij}x_ip_i x_j\frac{\partial}{\partial x_j} - \sum_{1 \leq j \leq n, 1 < i \leq n}\Omega_{ij}p_ix_i p_j\frac{\partial}{\partial p_j}.
	$$
	Note that the Poisson spray $X$ is complete and its flow has no loops. Moreover, $\overline{\omega}_t$ is non-degenerate since $\overline{\omega}_t^n = n \bigwedge\limits_{1\leq i\leq n} dp_i \wedge dx_i$ is a volume form.
\end{example}


%%%%%%%%%%%%%%%%%%%%%%%%%%
\section{Cluster Algebras}

Let $\tilde B=(B_{ij})$ be an $m\times n$ integer matrix with $m\ge n$.  
Write $B$ for the upper $n\times n$ submatrix of $\tilde B$ and assume $B$ is skew-symmetrizable, i.e. there exists an integer diagonal matrix $D=\diag(d_1,\ldots,d_n)$ with each $d_i>0$ so that $DB$ is skew-symmetric. 
Such an $m\times n$ matrix $\tilde B$ with skew-symmetrizable principal submatrix $B$ is called an \emph{exchange matrix}.
We fix a skew-symmetrizing matrix $D$.
An $m\times m$ matrix $\Omega=(\Omega_{ij})$ is \emph{compatible} with $(\tilde B,D)$ if $\tilde B^T\Omega=[D\ 0]$, where $0$ here denotes an $n\times(m-n)$ matrix with all zero entries.

For $1\le k\le n$, define the \emph{mutation of $\tilde B$ in direction $k$} by $\mu_k\tilde B=(B'_{ij})$, where
\[B'_{ij}=\begin{cases}-B_{ij} & \text{if $i=k$ or $j=k$;}\\ B_{ij}+[B_{ik}]_+B_{kj}+B_{ik}[-B_{kj}]_+ & \text{otherwise.}\end{cases}\]
Above we used the notation $[a]_+=\max\{a,0\}$.

To record the iteration of mutations we introduce the $n$-regular rooted tree $\TT_n$ with root vertex $t_0$ and the $n$ edges emanating from each vertex labeled by the set $\{1,\ldots,n\}$.
In particular, each vertex $t$ of $\TT_n$ is uniquely determined by a sequence of indices specifying the edge labels along the unique path from $t_0$ to $t$.
Fix an initial $m\times n$ exchange matrix $\tilde B_{t_0}$ and assign exchange matrices $\tilde B_t$ to the vertices of $\TT_n$ so that $\tilde B_{t'}=\mu_k\tilde B_t$ whenever $t$ and $t'$ are joined by an edge labeled by $k$.
The collection $\{\tilde B_t\}_{t\in\TT_n}$ is called the \emph{mutation pattern} generated by $\tilde B_{t_0}$ and any two exchange matrices $\tilde B_t$, $\tilde B_{t'}$ for $t,t'\in\TT_n$ are said to be \emph{mutation equivalent}.

\begin{example}
  Given an $n\times n$ skew-symmetrizable matrix $B$, let $\tilde B_{t_0}=\tilde B_{prin}$ denote the $2n\times n$ matrix with principal submatrix $B$ and lower $n\times n$ submatrix given by the $n\times n$ identity matrix $I_n$.
  %In this case a seed $\Sigma_{prin}=(\bfx_{prin},\tilde B_{prin})$ is said to have \emph{principal coefficients}.
  Given $\tilde B_t$ mutation equivalent to $\tilde B_{t_0}$, the lower $n\times n$ submatrix $C_t$ of $\tilde B_t$ has the following remarkable \emph{sign-coherence} property: each column of $C_t$, known as a $\bfc$-vector, has either all non-negative entries or all non-positive entries \cite{fomin-zelevinsky4,nakanishi-zelevinsky,gross-hacking-keel-kontsevich}.
  Thus each exchange matrix $\tilde B_t$ mutation equivalent to $\tilde B_{t_0}$ admits a collection of \emph{tropical signs} $\varepsilon_{k,t}\in\{\pm1\}$, where $\varepsilon_{k,t}=1$ if the entries in the $k$-th column of $C_t$ are non-negative and $\varepsilon_{k,t}=-1$ if the entries in the $k$-th column of $C_t$ are non-positive. 
\end{example}

\begin{remark}
  Given an arbitrary exchange matrix $\tilde B$, we may construct a corresponding prinicaplized exchange matrix $\tilde B_{prin}$ from the principal submatrix $B$.
  Then any exchange matrix $\tilde B_t$ mutation equivalent to $\tilde B$, we associate the same tropical signs $\varepsilon_{k,t}$ to the columns of $\tilde B_t$.
\end{remark}

For $1\le k\le n$ and a sign $\varepsilon\in\{\pm1\}$, let $E_{k,\varepsilon}$ be the $m\times m$ matrix with entries
\[E_{ij}=\begin{cases}\delta_{ij} & \text{if $j\ne k$;}\\ -1 & \text{if $i=j=k$;}\\ [-\varepsilon B_{ik}]_+ & \text{if $i\ne j=k$.}\end{cases}\]



Let $\cF$ be an extension field of $\QQ$ of transcendence degree $m$.   
A \emph{seed} in $\cF$ is a triple $\Sigma=(\bfx,\tilde B)$, where $\bfx=(x_1,\ldots,x_m)$ is a transendence basis of $\cF$ over $\QQ$ called the \emph{cluster} with entries called \emph{cluster variables} and $\tilde B$ is an $m\times n$ exchange matrix.
For $1\le k\le n$, define the \emph{mutation of $\Sigma$ in direction $k$} by $\mu_k\Sigma=(\mu_k\bfx,\mu_k\tilde B)$, $\mu_k\bfx=(x'_1,\ldots,x'_m)$ is given by the \emph{exchange relation}
\begin{equation}
  \label{eq:exchange relations}
  x'_i=\begin{cases} x_i & \text{if $i\ne k$;}\\ \frac{1}{x_k}\left(\prod\limits_{i=1}^m x_i^{[B_{ik}]_+}+\prod\limits_{i=1}^m x_i^{[-B_{ik}]_+}\right) & \text{if $i=k$;}\end{cases}
\end{equation}
Observe that seed mutation is involutive, i.e. $\mu_k(\mu_k\Sigma)=\Sigma$.
Given an initial seed $\Sigma_{t_0}=(\bfx_{t_0},\tilde B_{t_0})$ we label seeds $\Sigma_t=(\bfx_t,\tilde B_t)$ which are mutation equivalent to $\Sigma_{t_0}$ by vertices of $\TT_n$ as above, here we write $\bfx_t=(x_{1;t},\ldots,x_{m;t})$.
The resulting collection $\{\Sigma_t\}_{t\in\TT_n}$ is known as the \emph{exchange pattern} generated by $\Sigma_{t_0}$.
\begin{definition}
  Let $\Sigma$ be a seed in $\cF$.  The \emph{cluster algebra} $\cA(\Sigma)$ is the $\ZZ$-subalgebra of $\cF$ generated by all cluster variables $x_{i;t}$ from seeds $\Sigma_t$ mutation equivalent to $\Sigma_{t_0}$.
\end{definition}
By iterating the exchange relations we appear to get elements of $\QQ(x_1,\ldots,x_m)\subset\cF$, that is rational functions in $x_1,\ldots,x_m$.  
The following result of Fomin and Zelevinsky known as ``the Laurent phenomenon'' shows that the cluster variables always take on a much simpler form.
\begin{theorem}
  \cite{fomin-zelevinsky1}
  Let $\Sigma$ be a seed in $\cF$ and $\Sigma_t$ any seed in the exchange pattern generated by $\Sigma$.
  Each cluster variable $x_{i;t}$ of $\Sigma_t$ is an element of the subring $\ZZ[x_1^{\pm1},\ldots,x_m^{\pm1}]\subset\cF$.
\end{theorem}

In fact, the situtation is even better:\footnote{do we need this?} the initial cluster Laurent expansions of all cluster variables have positive integer coefficients.
\begin{theorem}\cite{lee-schiffler, gross-hacking-keel-kontsevich}
  Let $\Sigma$ be a seed in $\cF$ and $\Sigma'\sim\Sigma$.  Each cluster variable $x'_i$ of $\Sigma'$ is an element of the subsemiring $\ZZ_{\ge0}[x_1^{\pm1},\ldots,x_m^{\pm1}]\subset\cF$. 
\end{theorem}
For $x'_i$ a cluster variable from a seed $\Sigma'\sim\Sigma$, we write $x'_i(\bfx)$ when we wish to emphasize that $x'_i$ should be thought of as a function of the cluster variables in $\bfx=(x_1,\ldots,x_m)$.

\subsection{The Cluster Manifold and Compatible Poisson Structures}
Fix the field $\FF=\RR$ or $\FF=\CC$.
For an $m\times n$ exchange matrix $\tilde B$, define the \emph{cluster chart} $\Spec(\FF[x_1^{\pm1},\ldots,x_m^{\pm1}])$.
Observe that $\Sigma=(\bfx,\tilde B)$ is a seed in the field of rational functions on this cluster chart and thus we denote it by $L_\Sigma$.
Then the exchange relation \eqref{eq:exchange relations} provides a birational transformation between the cluster charts $\varphi_{\Sigma,\mu_k\Sigma}:L_\Sigma\to L_{\mu_k\Sigma}$ for $1\le k\le n$.
By composing these \emph{elementary transition maps} for neighboring seeds we get a birational transformation between $\varphi_{\Sigma,\Sigma'}:L_\Sigma\to L_{\Sigma'}$ for any seeds $\Sigma\sim\Sigma'$.

Given any seed $\Sigma$, the transition maps above define the \emph{cluster manifold} $M=M(\Sigma)=\bigcup\limits_{\Sigma'\sim\Sigma}L_{\Sigma'}$.
By construction we have $\cA(\Sigma)\subset C^\infty(M)$ and any Poisson structure on $\cA(\Sigma)$ naturally extends to give a Poisson structure on $C^\infty(M)$.

\begin{definition}
  A Poisson structure $\{\cdot,\cdot\}:\cA(\Sigma)\times\cA(\Sigma)\to\cA(\Sigma)$ is \emph{compatible} with the cluster algebra structure if, for each seed $\Sigma'\sim\Sigma$, the cluster variables in $\bfx'$ are \emph{log-canonical} with respect to $\{\cdot,\cdot\}$.
  That is, there exists a skew-symmetric integer \emph{coefficient matrix} $\Omega'=(\Omega'_{ij})$ so that 
  \begin{equation}
    \label{eq:log-canonical bracket}
    \{x'_i,x'_j\}=\Omega'_{ij}x'_ix'_j
  \end{equation}
  for $1\le i,j\le m$.
\end{definition}
\begin{remark}
  Suppose the cluster variables of a seed $\Sigma=(\bfx,\tilde B)$ are log-canonical with respect to a Poisson bracket $\{\cdot,\cdot\}:\cA(\Sigma)\times\cA(\Sigma)\to\cA(\Sigma)$ with coefficient matrix $\Omega$.
  Then the compatibility of $\{\cdot,\cdot\}$, together with the exchange relations, imposes the compatibility condition $\tilde B^T\Omega=[D\ 0]$ (see \cite{berenstein-zelevinsky,gekhtman-shapiro-vainshtein} for details).
\end{remark}

\begin{theorem}
  \cite{gekhtman-shapiro-vainshtein}
  Suppose the $m\times n$ exchange matrix $\tilde B$ of a seed $\Sigma$ has full rank.  Then there exists a Poisson structure $\Omega$ compatible with the cluster structure on $\cA(\Sigma)$.
\end{theorem}


%%%%%%%%%%%%%%%%%%%%%%%%%%%%%%%%%%%%%%
\section{Cluster Symplectic Groupoids}
Let $\Sigma=(\bfx,\tilde B,\Theta)$ be a graded seed and assume there exists a compatible Poisson structure on $L_\Sigma$ with coefficient matrix $\Omega=(\Omega_{ij})$.  
%Write $\pi$ for the corresponding Poisson bivector on the cluster manifold $M$, i.e.\ in local coordinates on $L_\Sigma$ we have $\pi=\sum\limits_{i>j}\Omega_{ij}x_ix_j\frac{\partial}{\partial x_i}\wedge\frac{\partial}{\partial x_j}$.
In this section, we give an integration to a symplectic groupoid $G(\Sigma)$ of the Poisson structure on a cluster manifold $M(\Sigma)$.  

We build the cluster symplectic groupoid $G(\Sigma)\rightrightarrows M(\Sigma)$ by gluing together local groupoid charts $G_{\Sigma'}\rightrightarrows L_{\Sigma'}$, $\Sigma'\sim\Sigma$, along transition maps which lift the cluster mutations used to glue cluster charts of $M(\Sigma)$.
This process is carried out in three steps:
\begin{itemize}
  \item first, we show that the action groupoids $(\FF^*)^m\times L_{\Sigma'}\rightrightarrows L_{\Sigma'}$ over each cluster chart admit a gluing which lifts the cluster mutation;
  \item second, we define maps $T^*L_{\Sigma'}\to(\FF^*)^m\times L_{\Sigma'}$ along which we pullback the groupoid structure to obtain symplectic groupoids integrating a compatible Poisson structure on $L_{\Sigma'}$;
  \item finally, we define transition maps between the symplectic groupoids $G_{\Sigma'}=T^*L_{\Sigma'}$ which lift the cluster mutations.
\end{itemize}

There is a natural action groupoid structure $(\FF^*)^m\times L_\Sigma\rightrightarrows L_\Sigma$ with source map $\alpha$ being the natural projection and target map given by the Hadamard product
\[\beta(\bfs,\bfx)=\bfs\circ\bfx,\quad \bf\circ\bfx=(s_1x_1,\ldots,s_mx_m),\]
i.e.\ given by the natural action of $(\FF^*)^m$ on $L_\Sigma$.

\erase{
Given any seed $\Sigma'\sim\Sigma$, define a map $\mu_{\Sigma',\Sigma}:(\FF^*)^r\times L_\Sigma\to(\FF^*)^r\times L_\Sigma'$ by $\mu_{\Sigma',\Sigma}(\bfs,\bfx)=(\bfs',\bfx')$, where $\bfx'(\bfx)=\big(x'_1(\bfx),\ldots,x'_m(\bfx)\big)$ and $\bfs'(\bfs,\bfx)=(s'_1(\bfs,\bfx),\ldots,s'_m(\bfs,\bfx))$ is given by $s'_i(\bfs,\bfx)=\frac{x'_i(\bfs\circ\bfx)}{x'_i(\bfx)}$.
\begin{theorem}
  For any three seeds $\Sigma\sim\Sigma'\sim\Sigma''$, we have $\mu_{\Sigma'',\Sigma'}\mu_{\Sigma',\Sigma}=\mu_{\Sigma'',\Sigma}$ and hence the local groupoid charts glue to give a groupoid over the cluster manifold $M(\Sigma)$.
\end{theorem}
\begin{proof}
  By induction, it suffices to prove the claim when $\Sigma''=\mu_k\Sigma'$ for some $k$.
  In this case, we have $x''_i(\bfx)=x'_i(\bfx)$ and thus $s''_i(\bfs,\bfx)=s'_i(\bfs,\bfx)$ for $i\ne k$.
  Observe that the definition of $\mu_{\Sigma',\Sigma}$ gives $\bfs'(\bfs,\bfx)\circ\bfx'(\bfx)=\bfx'(\bfs\circ\bfx)$ and the definition of $\mu_k$ gives $\bfx''(\bfx'(\bfx))=\bfx''(\bfx)$.
  It then immediately follows from the definition of $\mu_{\Sigma'',\Sigma'}\mu_{\Sigma',\Sigma}$ that we have
  \[s''_k(\bfs'(\bfs,\bfx),\bfx'(\bfx))=\frac{x''_k(\bfs'(\bfs,\bfx)\circ\bfx'(\bfx))}{x''_k(\bfx'(\bfx))}=\frac{x''_k(\bfx'(\bfs\circ\bfx))}{x''_k(\bfx'(\bfx))}=\frac{x''_k(\bfs\circ\bfx)}{x''_k(\bfx)}=s''_k(\bfs,\bfx).\]
\end{proof}}%end erase

Let $G_\Sigma=T^*L_\Sigma$ denote the cotangent bundle of $L_\Sigma$.
Write $\bfp=(p_1,\ldots,p_m)$ for the cotangent coordinates of $G_\Sigma$.
Define a map $\rho_\Sigma:G_\Sigma\to(\FF^*)^m\times L_\Sigma$ by $\rho_\Sigma(\bfx,\bfp)=(\bfs(\bfx,\bfp),\bfx)$, with $s_i(\bfx,\bfp)=e^{\sum_j\Omega_{ij}x_jp_j}$.

%Define the map $\beta:G_\Sigma\to L_\Sigma$ by 
%\begin{equation}
%  \label{eq:cluster groupoid target map}
%  \beta(\bfx,\bfp)=(s_1x_1,\ldots,s_nx_n),\quad\text{where}\quad s_i:=e^{\sum_j\Omega_{ij}x_jp_j}.
%\end{equation}
\begin{theorem}
  \label{th:cluster groupoid}
  The groupoid structure on $(\FF^*)^m\times L_\Sigma$ pulls back to a groupoid structure on the manifold $G_\Sigma$ with source map the natural projection, target map $\beta\circ\rho_\Sigma$, multiplication given by
  \[(\bfx,\bfp)\cdot\big((\beta\circ\rho_\Sigma)(\bfx,\bfp),\bfp'\big)=(\bfx,\bfp''),\quad p''_i=s_i(\bfx,\bfp)p'_i+p_i,\]
  inversion given by
  \[(\bfx,\bfp)^{-1}=\big(\beta(\bfx,\bfp),\bfp'),\quad p'_i=-s_i(\bfx,\bfp)^{-1}p_i,\]
  and identity map given by $1_\bfx=(\bfx,\boldsymbol{0})$.
  %Moreover, the 2-form $\omega=\alpha^*(\pi^{-1})-\beta^*(\pi^{-1})$ defines a symplectic structure on $G_\Sigma$.
\end{theorem}
%\begin{proof}
%  It is clear that $1_\bfx$ gives the identity map for this multiplication and that the inversion map satisfies 
%  \[(\bfx,\bfp)\cdot(\bfx,\bfp)^{-1}=1_\bfx=(\bfx,\bfp)^{-1}\cdot(\bfx,\bfp)\]
%  for all $(\bfx,\bfp)\in G_\Sigma$.
%
%  It remains to check associativity of the multiplication.
%  Fix an element $(\bfx,\bfp)\in G_\Sigma$.
%  Consider $(\bfx',\bfp'),(\bfx'',\bfp'')\in G_\Sigma$ with $\bfx'=\beta(\bfx,\bfp)$ and $\bfx''=\beta(\bfx',\bfp')$.
%  Then we have
%  \[\big((\bfx,\bfp)\cdot(\bfx',\bfp')\big)\cdot(\bfx'',\bfp'')=(\bfx,\bfp'''),\quad p'''_i=e^{\sum_j\Omega_{ij}x_j(s_jp'_j+p_j)}p''_i+e^{\sum_j\Omega_{ij}x_jp_j}p'_i+p_i.\]
%  On the other hand we have
%  \[(\bfx,\bfp)\cdot\big((\bfx',\bfp')\cdot(\bfx'',\bfp'')\big)=(\bfx,\bfp'''),\quad p'''_i=e^{\sum_j\Omega_{ij}x_jp_j}(e^{\sum_j\Omega_{ij}x'_jp'_j}p''_i+p'_i)+p_i\]
%  and thus associativity holds.
%\end{proof}

Write $\mu_k\Sigma=(\bfx',\tilde B')$.  
%Let $\bfp'=(p'_1,\ldots,p'_n)$ denote the cotangent coordinates of $G_{\mu_k\Sigma}$.
Define a map from $G_\Sigma$ to $G_{\mu_k\Sigma}$, which we also denote $\mu_{\Sigma,\mu_k\Sigma}$, as follows:
\begin{equation}
  \label{eq:groupoid gluing map}
  \mu_{\Sigma,\mu_k\Sigma}(\bfx,\bfp)=(\bfx'(\bfx),\bfp'(\bfx,\bfp)),\quad \bfp'(\bfx,\bfp)=(p'_1(\bfx,\bfp),\ldots,p'_m(\bfx,\bfp)),\quad p'_\ell(\bfx,\bfp)=\frac{x_\ell p_\ell +[\varepsilon_k b_{\ell k}]_+ x_k p_k +\frac{b_{\ell k}}{d_k}\ln\left(\frac{Q_k(\bfs\circ\bfx)}{Q_k(\bfx)}\right)}{x'_\ell(\bfx)},
\end{equation}
where $\varepsilon_k$ denotes the tropical sign for the seed $\Sigma$.

\begin{lemma}
  Let $\bfx',\bfp',\Omega'$ be obtained from $\bfx,\bfp,\Omega$ by mutation in direction $k$.
  For any index $1\le i\le m$ with $i\ne k$, we have $\sum\limits_{j=1}^m \Omega'_{ij} x'_j p'_j = \sum\limts_{j=1}^m \Omega_{ij} x_j p_j$ and $\sum\limits_{j=1}^m \Omega'_{kj} x'_j p'_j=\ln\left\frac{x'_k(\bfs\circ\bfx)}{x'_k(\bfx)}$.
\end{lemma}
\begin{proof}
  The mutation operation for groupoid charts is more naturally written in vector form as
  \[\bfx'\circ\bfp'=E_{k,\varepsion_k}(\bfx\circ\bfp)+\frac{1}{d_k}\ln\left(\frac{Q_k(\bfs\circ\bfx)}{Q_k(\bfx)}\bfb_k.\]
  This gives rise to the identity
  \begin{align}
    \Omega'(\bfx'\circ\bfp')
    \nonumber &=\Omega' E_{k,\varepsion_k}(\bfx\circ\bfp)+\frac{1}{d_k}\ln\left(\frac{Q_k(\bfs\circ\bfx)}{Q_k(\bfx)}\Omega'\bfb_k\\
    \label{eq:groupoid transition} &=E_{k,\varepsion_k}^T\Omega (\bfx\circ\bfp)+\ln\left(\frac{Q_k(\bfs\circ\bfx)}{Q_k(\bfx)}\bfe_k.
  \end{align}
  By the structure of $E_{k,\varepsilon_k}$, the euqality $\sum\limits_{j=1}^m \Omega'_{ij} x'_j p'_j = \sum\limts_{j=1}^m \Omega_{ij} x_j p_j$ for $i\ne k$ immediately follows.

  Equation~\ref{eq:groupoid transition} also gives the identity
  \begin{align*}
    \sum\limits_{j=1}^m \Omega'_{kj} x'_j p'_j
    &=-\sum\limits_{j=1}^m \Omega_{kj} x_j p_j+\ln\left(\bfs^{\bfb_k^+}\right)+\ln\left(\frac{Q_k(\bfs\circ\bfx)}{Q_k(\bfx)}\\
    &=-\sum\limits_{j=1}^m \Omega_{kj} x_j p_j+\ln\left(\frac{x'_k(\bfs\circ\bfx)x_k(\bfs\circ\bfx)}{x'_k(\bfx)x_k(\bfx)}\\
    \ln\left\frac{x'_k(\bfs\circ\bfx)}{x'_k(\bfx)}.
  \end{align*}
\end{proof}

\begin{lemma}
  The mutation of cluster groupoid charts is involutive.
\end{lemma}
\begin{proof}
  
\end{proof}


%%%%%%%%%%%%%%%%%%%%%%%%%%%%%%%%%%%%%%%%%%%%
\section{Totally Positive Cluster Manifolds}
In this section we show that the totally nonnegative part $M_{\ge0}(\Sigma)$ of a cluster manifold is a manifold with corners in the sense of \cite{MR3077259}.  
Moreover, we show that the nonnegative cluster manifold is a union of symplectic leaves for any compatible Poisson structure on $\cA(\Sigma)$.  
The symplectic leaves of $M_{\ge0}(\Sigma)$ are naturally labelled by compatible subsets of cluster variables, where the number of cluster variables in the labeling set determines the corank of the symplectic leaf.
Here there is a unique dense symplectic leaf and the boundary of $M_{\ge0}(\Sigma)$ is again a union of symplectic leaves of lower dimension where the Poisson structure degenerates.

\begin{theorem}
  Let $\Sigma$ be a seed.  
  The 1-skeleton of $M_{\ge0}(\Sigma)$ given by 0-dimensional and 1-dimensional symplectic leaves identifies with the exchange graph of $\cA(\Sigma)$.  
  Moreover, if $\Sigma$ is a seed of finite-type, then $M_{\ge0}(\Sigma)$ provides a realization of the generalized associahedron with the same Cartan type as $\Sigma$.
\end{theorem}
\begin{proof}
  The 0-dimensional symplectic leaves correspond to the vanishing of all cluster variables from a seed mutation equivalent to $\Sigma$.  
  Then a 1-dimensional symplectic leaf whose boundaries correspond to seeds $\Sigma'$ and $\Sigma''$ exactly corresponds to the non-vanishing of exchangable cluster variables $x'_k$ and $x''_k$.
  But this is exactly the exchange graph of $\cA(\Sigma)$.

  When $\Sigma$ is of finite-type, the realization of $M_{\ge0}(\Sigma)$ as a simplicial complex, given by taking symplectic leaves as cells, is naturally dual to the cluster complex of $\cA(\Sigma)$, i.e. $M_{\ge0}(\Sigma)$ identifies with the associated generalized associahedron.
\end{proof}

%%%%%%%%%%%%%%%%%%%%%%%%%%%%%%%%%%%%%%%%%%%%%%%%%%%%%%%%%%%%%%%%
\section{Symplectic Topology of the Nonnegative Cluster Groupoid}
Let $\cG_{\ge0}(\Sigma)$ denote the symplectic groupoid over $M_{\ge0}(\Sigma)$.  
In this section we introduce a Poisson spray which may be used to construct $\cG_{\ge0}(\Sigma)$ and apply a Moser argument to show that up to symplectomorphism $\cG_{\ge0}(\Sigma)$ can be identified with the natural symplectic structure on the cotangent bundle $T^*M_{\ge0}(\Sigma)$.

Let $\cA(\Sigma)$ be a cluster algebra of rank $n$ generated by the seed $\Sigma=(\bfx,\tilde B)$, and we assume there exists a compatible Poisson structure on $L_\Sigma$ with coefficient matrix $\Omega=(\Omega_{ij})$. That is,  $\tilde B^T\Omega=[D\ 0]$, where $D$ is a skew-symmetrizing matrix for the upper $n\times n$ submatrix $B$ of $\tilde B$, and $L_\Sigma$ is equipped with the Poisson structure
\begin{equation}
	\pi=\sum\limits_{1 \leq j \leq n, j < i \leq n}\Omega_{ij}x_ix_j\frac{\partial}{\partial x_i}\wedge\frac{\partial}{\partial x_j}.
\end{equation}
By Remark~\ref{remark:PoiSpLoc}, we may choose the Poisson spray $X$ on $L_\Sigma$:
\begin{equation}
	X = \sum\limits_{1 \leq j \leq n, j < i \leq n}\Omega_{ij}x_ip_i x_j\frac{\partial}{\partial x_j} \quad - \sum\limits_{1 \leq j \leq n, j < i \leq n}\Omega_{ij}p_ix_i p_j\frac{\partial}{\partial p_j}.
\end{equation}

\bibliographystyle{hyperamsplain}
\bibliography{cluster_symplectic}

\end{document}
