\documentclass{amsart}
\usepackage{amsmath,amsfonts,amssymb,latexsym}
\usepackage[margin=1in]{geometry}
\usepackage{color}
\usepackage[pagebackref, bookmarks=true, bookmarksopen=true, bookmarksdepth=3,bookmarksopenlevel=2, colorlinks=true, linkcolor=blue, citecolor=blue, filecolor=blue, menucolor=blue, urlcolor=blue]{hyperref}


\newtheorem{theorem}{Theorem}[section]
\newtheorem{corollary}[theorem]{Corollary}
\newtheorem{definition}[theorem]{Definition}
\newtheorem{lemma}[theorem]{Lemma}
\newtheorem{question}[theorem]{Question}
\newtheorem{proposition}[theorem]{Proposition}
\newtheorem{remark}[theorem]{Remark}
\newtheorem{example}[theorem]{Example}

\numberwithin{equation}{section}

\newcommand{\bfa}{\mathbf{a}}
\newcommand{\bfb}{\mathbf{b}}
\newcommand{\bfc}{\mathbf{c}}
\newcommand{\bfe}{\mathbf{e}}
\newcommand{\bfp}{\mathbf{p}}
\newcommand{\bfq}{\mathbf{q}}
\newcommand{\bfs}{\mathbf{s}}
\newcommand{\bft}{\mathbf{t}}
\newcommand{\bfx}{\mathbf{x}}
\newcommand{\bfy}{\mathbf{y}}

\newcommand{\cA}{\mathcal{A}}
\newcommand{\cF}{\mathcal{F}}
\newcommand{\cG}{\mathcal{G}}
\newcommand{\cH}{\mathcal{H}}
\newcommand{\cD}{\mathcal{D}}
\newcommand{\cO}{\mathcal{O}}
\newcommand{\cP}{\mathcal{P}}
\newcommand{\cX}{\mathcal{X}}
\newcommand{\cT}{\mathcal{T}}
\newcommand{\cU}{\mathcal{U}}

\newcommand{\CC}{\mathbb{C}}
\newcommand{\FF}{\mathbb{F}}
\newcommand{\kk}{\Bbbk}
\newcommand{\QQ}{\mathbb{Q}}
\newcommand{\RR}{\mathbb{R}}
\newcommand{\TT}{\mathbb{T}}
\newcommand{\ZZ}{\mathbb{Z}}

\newcommand{\diag}{\operatorname{diag}}
\newcommand{\Hom}{\operatorname{Hom}}
\newcommand{\Li}{\operatorname{Li}}
\renewcommand{\max}{\operatorname{max}}
\newcommand{\Spec}{\operatorname{Spec}}

\newcommand{\rra}{\rightrightarrows}

\newcommand{\erase[1]}{{}}

\title{Symplectic Groupoids for Cluster Manifolds}
\author{Songhao Li}
\address[Songhao Li]{University of Notre Dame}
\email{sli19@nd.edu}

\author{Dylan Rupel}
\address[Dylan Rupel]{University of Notre Dame}
\email{drupel@nd.edu}

\begin{document}
\begin{abstract}
  We construct a source simply-connected symplectic groupoid integrating a log-canonical Poisson structure on a cluster manifold.
\end{abstract}
\maketitle
Outline
\begin{enumerate}
	\item Intro to cluster algebra and compatible Poisson structures 
	\item Intro to Poisson manifolds, symplectic groupoid and Poisson spray
	\item Cluster symplectic groupoid
	\item Totally positive cluster manifolds (definition of manifolds with corners [check D Joyce], associahedron of type A and generalized associahedron)
	\item Examples
\end{enumerate}

%%%%%%%%%%%%%%%%%%%%%%
\section{Introduction}
This is the first of a series of papers whose aim is to implement Weinstein's program of geometric quantization for Poisson manifolds \cite{MR1104934} in the case of cluster Poisson structures. In this first paper, we construct the symplectic groupoids for both the cluster $\cA$-varieties and the cluster $\cX$-varieties. 

The cluster $\cA$-varieties are the geometric realization of cluster algebras defined by Fomin and Zelevinsky \cite{MR1887642} as the culmination of their study of total positivity and canonical bases for algebraic groups. Many varieties arising in Lie theory, e.g.\ Grassmannians and double Bruhat cells, are examples of $\cA$-varieties \cite{berenstein-fomin-zelevinsky,scott,gekhtman-shapiro-vainshtein,williams}. The cluster $\cA$-varieties are often endowed with a class of compatible Poisson structures in the sense of Gekhtman, Shapiro, and Vainshtein \cite{MR2683456}.

The cluster $\cX$-varieties were introduced by Fock and Goncharov \cite{MR2470108}\footnote{Find earlier reference} in their study of higher Teichmuller space. The $\cX$-varieties are always endowed with a canonical Poisson structure. If a $\cA$-variety carries a compatible Poisson structure, then the natural map from this $\cA$-variety to the corresponding $\cX$-variety is Poisson.

On the other hand, quantization have different meanings. In the case of cluster algebras, the quantum $\cA$-varieties \cite{berenstein-zelevinsky} and the quantum $\cX$-varieties \cite{MR2470108} are both concrete examples of deformation quantization of Poisson manifolds \cite{MR2062626}. Geometric quantization for symplectic manifolds takes one step further by constructing a Hilbert space, but this requires that the cohomology class of the symplectic 2-form to be integral. See e.g. \cite{MR1806388}.
The notion of symplectic groupoids was introduced independently by Weinstein \cite{MR866024}, Karas\"{e}v \cite{MR1008479} and Zakrzewski \cite{MR1081010, MR1081011}. We list some important examples of symplectic groupoids: the Lie-Poisson structures \cite{MR996653}, the Drinfeld double of Poisson Lie groups \cite{MR1054741}, the double Bruhat cells \cite{LuM16}, the blow-up groupoid of log symplectic manifolds \cite{MR3214314}, and the cluster $\cX$-varieties \cite{MR2470108}.

Weinstein's original motivation was to find a geometric quantization schema for Poisson manifolds \cite{MR1104934}. This has been successfully implemented for a handful of examples \cite{MR2417440, MR2925830}. The concrete nature of cluster algebra makes it an ideal testing ground for groupoid quantization. Indeed, this idea was implicitly implemented by Fock and Goncharov \cite{MR2470108} in the case of $\cX$-varieties.

In this paper, we take the first step towards the groupoid quantization for both $\cA$-varieties and $\cX$-varieties, by constructing their symplectic groupoids. The paper is organized as follows.

\begin{itemize}
	\item In Section~\ref{sec: local}, we build the symplectic groupoid in each local chart of a cluster variety. This is essentially the construction of local symplectic groupoids by Poisson sprays \cite{MR2900786, CMS17}, but for a log-canonical Poisson structure, there is a clever choice of the Poisson spray such that the construction yields the global source-simply-connected symplectic groupoid $\cG$. We also recall the symplectic double $\cD$ as in \cite{MR2470108}, albeit with slightly different notations to accommodate the fact that we consider the non-square exchange matrices in this paper. The symplectic double is a source-connected symplectic groupoid of the log-canonical Poisson structure and admits a groupoid covering $\kappa: \cG \to \cD$.
	\item In Section~\ref{sec:cluster}, we give the explicit formulae for the cluster groupoid mutation $\tilde{\mu}: \cG \to \cG'$. The first step is to decompose the cluster mutation $\mu: L \to L'$ into two maps $\varphi^1$ and $\tau$, where $\varphi^1$ is the time-$1$-flow of the Euler dilogarithm function and $\tau$ is a natural transformation of log-canonical coordinates. See \cite{MR2567745} for the case of $\cX$-varieties and \cite{MR3691969} for the case of $\cA$-varieties. We then lift $\varphi^1$ and $\tau$ to groupoid maps $\tilde{\varphi}^1$ and $\tilde{\tau}$, and the composition $\tilde{\mu} = \tilde{\tau} \circ \tilde{\varphi}^1$ is the cluster groupoid mutation.
\end{itemize}

\bigskip


We use the following notations. For vectors and matrices,
\begin{itemize}
	\item boldface letters denote vectors, e.g. $\bfx = (x_1, \ldots, x_m)$;
	\item (Hadamard product) for two vector $\bfx$ and $\bfy$, $\bfx \circ \bfy = (x_1y_1, \ldots, x_my_m)$;
	\item for a vector $\bfx$ and a scalar $t$, $\bfx^t = (x_1^t, x_m^t)$;
	\item for two vector $\bfx$ and $\bfa$, $\bfx^\bfa = \prod_{i=1}^m x_i^{a_i}$;
	\item for a matrix $B = (B_{ij})$, $B_{k}$ denotes the kth column vector of $B$.
\end{itemize}
For a manifold $M$,
\begin{itemize}
	\item we denote the tangent bundle by $TM$, the sheaf of vector fields by $\cT_M$, and the sheaf of bi-vectors by $\cT^2_M$;
	\item we denote the cotangent bundle by $T^*M$, the k-forms by $\Omega^k(M)$.
\end{itemize}


%%%%%%%%%%%%%%%%%%%%%%%%%%%%%%%%%%%%%%%%%%%%%%%%%%%%%%%%%%%%%%%%%%
\section{Symplectic groupoids of log canonical Poisson structures} \label{sec: local}

%In this section, we will use the idea of Poisson spray \cite{MR2900786} to construct the source-simply-connected sympletic groupoid of a log-canonical Poisson structure.
%There is another construction of a source-connected symplectic groupoid, called the symplectic double \cite{MR2470108}.
%We give an explicit formula for the groupoid morphism from the source-simply-connected symplectic groupoid to the symplectic double.
%We recall the natural Poisson map between two compatible log-canonical Poisson structures, which serve as the seeds for cluster $\cA$-varieties and $\cX$-varieties.

To introduce the notion of symplectic groupoids and Poisson sprays, we recall the equivalent notions of Poisson brackets and Poisson bi-vectors.
\begin{definition}
  Let $M$ be either a smooth manifold or a complex manifold.
  A \emph{Poisson structure} on $M$ is one of the two following equivalent structures:
  \begin{enumerate}
    \item a \emph{Poisson bracket}
      $$\{\cdot, \cdot\}: \cO_M \times \cO_M \to \cO_M$$
      which is a Lie bracket satisfying the Leibniz rule
      $$\{fg, h\} = f\{g,h\} + g\{f,h\};$$
    \item a \emph{Poisson bi-vector} $\pi \in \cT^2_M$ such that $[\pi, \pi] = 0$, where $[\cdot, \cdot]$ is the Schouten-Nijenhuis bracket.
  \end{enumerate}	
\end{definition}

The two notions are related by the formula: $\{f, g\} = \pi (df \otimes dg)$ for $f, g\in \cO_\cX$.
The pair $(M, \pi)$, or equivalently $(M, \{\cdot,\cdot\})$, is called a \emph{Poisson manifold}.
A bi-vector $\pi \in \cT^2_M$ is called \emph{non-degenerate} if the bundle map
	\[
		\pi^\flat: T^*M \to TM, \qquad \alpha \mapsto \iota_\alpha \pi
	\]
is invertible. The inverse of the bundle map defines a non-degenerate 2-form $\omega \in \Omega^2(M)$. That is, the bundle map
	\[
		\omega^\sharp: TM \to T^*M, \qquad v \mapsto \iota_v \omega
	\]
is the inverse of $\pi^\flat$. In this case, we say $\omega^{-1} = \pi$. Note that non-degenerate Poisson structures are the same as symplectic structures, since $[\pi, \pi] = 0$ if and only if $d\omega = 0$.

\footnote{Define Poisson and symplectic groupoid maps.}

\begin{definition}
  A \emph{groupoid} $\cG\rightrightarrows M$ consists of two sets $\cG$ and $M$ with the following maps:
  \begin{enumerate}
    \item a surjective \emph{source map} $\alpha: \cG \to M$ and a surjective \emph{target map} $\beta: \cG \to M$;
    \item an injective identity map $\mathtt{1}: M \to \cG, \enskip x \mapsto \mathtt{1}_x$;
    \item an associative multiplication map $m: \cG {_\alpha \times_\beta} \cG \to \cG, \enskip (g, h) \mapsto gh$;
    \item and an involutive inversion map $\iota: \cG \to \cG, \enskip g \mapsto g^{-1}$;
  \end{enumerate}
  which satisfy the following properties:
  \begin{enumerate}
    \item $\alpha(\mathtt{1}_x) = \beta(\mathtt{1}_x) = x$;
    \item $\alpha(gh) = \alpha(h), \enskip \beta(gh) = \beta(g)$;
    \item $\alpha(g^{-1}) = \beta(g), \enskip \beta(g^{-1}) = \alpha(g)$;
    \item $(\mathtt{1}_x)^{-1} = \mathtt{1}_x$.
  \end{enumerate}
  A \emph{Lie groupoid} $\cG \rightrightarrows M$ has the following additional properties:
  \begin{enumerate}
    \item $\cG$ and $M$ are either smooth manifolds or complex manifolds;
    \item the source $\alpha: \cG \to M$ and the target $\beta: \cG \to M$ are surjective submersions;
    \item the multiplication map $m: \cG {_\alpha \times_\beta} \cG \to \cG$ is smooth;
    \item the inversion map $i: \cG \to \cG$ is smooth.
  \end{enumerate}
  A Lie groupoid $\cG \rightrightarrows M$ is \emph{source-connected} if $\alpha^{-1}(x)$ is connected for every $x \in M$; it is \emph{source-simply-connected} if $\alpha^{-1}(x)$ is connected and simply-connected for every $x \in M$.
\end{definition}

Given a Lie groupoid $\cG \rra M$, its $k$-nerve 
\[
	\cG^{(k)} = \{(g_1, g_2, \ldots, g_k) \in \cG^k ~|~ \beta(g_i) = \alpha(g_{i+1})\}
\]
is the set of $k$ composable elements. In particular, $\cG^{(2)} = \cG {_\alpha \times_\beta} \cG$, $\cG^{(1)} = \cG$ and $\cG^{(0)} = M$. The nerve of a Lie groupoid is naturally a simplicial manifold which carries a coboundary operator $\partial: \Omega^\bullet(\cG^{(k-1)}) \to \Omega^\bullet(\cG^{(k)})$. The first two operators are given as below:
\begin{align}
	& \partial: \Omega^\bullet(M) \to \Omega^\bullet(\cG), \qquad \mu \mapsto \alpha^*(\mu) - \beta^*(\mu) \\
	& \partial: \Omega^\bullet(\cG) \to \Omega^\bullet(\cG^{(2)}), \qquad \mu \mapsto \mathrm{pr}_1^*(\mu) - m^*(\mu) + \mathrm{pr}_2^*(\mu)
\end{align}
where $\mathrm{pr}_1: \cG^{(2)} \to \cG$ and $\mathrm{pr}_2: \cG^{(2)} \to \cG$ are the first and second projections. A differential form $\mu \in \Omega^\bullet(G)$ is called \emph{multiplicative} if $\partial \mu = 0$. Our main interest will be with Lie groupoids equipped with a multiplicative symplectic structure.
\begin{definition}
  A \emph{symplectic groupoid} is a Lie groupoid $\cG \rra M$ with a multiplicative symplectic structure $\omega \in \Omega^2(\cG)$.
  That is, $\mathrm{pr}_1^*(\omega)+\mathrm{pr}_2^*(\omega) = m^*(\omega)$ or equivalently the graph of the multiplication map $\Gamma_m = \{(g, h, gh) \in \cG \times \cG \times \cG\}$ is Lagrangian with repsect to the symplectic structure $\omega \oplus \omega \oplus -\omega$.
\end{definition}



For a symplectic groupoid $(\cG, \omega) \rra M$, there is a natural Poisson structure $\pi$ on $M$ such that $\alpha: (\cG, \omega) \to (M, \pi)$ and $\beta: (\cG, \omega) \to (M, -\pi)$ are Poisson maps.
Poisson manifolds that arise this way are called \emph{integrable}.
As a Lie groupoid, $\cG \rra M$ integrates the Poisson Lie algebroid $T^*M$ with the anchor map $\pi^\flat: T^*M \to TM$ and the Koszul bracket \cite{MR866024}; and as a Poisson groupoid, $(\cG, \omega) \rra (M, \pi)$ integrates the Lie bialgebroid $(T^*M, TM)$, where the tangent bundle $TM$ carries the standard Lie bracket on vector fields \cite{MR1262213}.
The integrability of Poisson manifolds, and more generally the integrability of Lie algebroids, is characterized by Crainic and Fernandes \cite{MR1973056, MR2128714}.

\begin{definition}
  Let $L$ be either $\RR^m$ or $\CC^m$ and write $\bfx = (x_1, \ldots, x_m)$ for a system of coordinates on $L$.
  A Poisson structure on $L$ is \emph{log-canonical} if
  $$
    \{x_i, x_j\} = \Omega_{ij} x_ix_j, \quad 1 \leq i,j \leq m \qquad \text{or equavalently} \quad
    \pi = \sum_{j < i} \Omega_{ij} x_ix_j\frac{\partial}{\partial x_i} \wedge \frac{\partial}{\partial x_j},
  $$
  for some skew-symmetric $m\times m$ matrix $\Omega = (\Omega_{ij})$.
\end{definition}

For both the cluster $\cA$-varieties \cite{MR2683456} and $\cX$-varieties \cite{MR2470108}, the compatible Poisson structures are log-canonical in local charts.
Using the results in \cite{MR2900786, CMS17}, we construct the source-simply-connected symplectic groupoid of a log-canonical Poisson structure by carefully chosing a Poisson spray. 
\begin{definition} 
  \cite{MR2900786}
  For a Poisson manifold $(M, \pi)$, a \emph{Poisson spray} is a vector field $X \in \cT_{T^*M}$ such that
  \begin{enumerate}
    \item for $(x,p) \in T^*M$ we have
      \[\tau_* X|_{(x,p)} = \pi^\flat(p),\]
      where $\tau: T^*M \to M$ is the bundle projection;
    \item $X$ is homogeneous of degree 1, i.e.
      \[(m_\lambda)_*(X) = \lambda X,\]
      where $m_\lambda: T^*M \to T^*M, \enskip (x,p) \mapsto (x,\lambda p)$ is the fiberwise scaling map.
  \end{enumerate}
\end{definition}

\begin{theorem} 
  \cite{MR2900786, CMS17} 
  \label{thm:poissp}
  For a smooth Poisson manifold $(M, \pi)$ with a Poisson spray $X \in \cT_{T^*M}$, a neighbourhood $U$ of the zero section of $T^*M$ is a local symplectic groupoid over $(M, \pi)$ with the following structures:
  \begin{enumerate}
    \item the source map $\alpha: U \to M$ is the bundle projection;
    \item the target map is
      \[\beta: U \to M, \qquad \beta = \tau \circ \varphi_X^1,\]
      where $\varphi_X^1: T^*M \to T^*M$ is the time-$1$-flow of $X$;
    \item the inverse map is
      \[\iota: U \to U, \qquad \iota = -\varphi_X^1;\]
    \item the multiplication $m: U {_\alpha \times_\beta} U \to U$ is defined as the solution of an ODE (see \cite{CMS17} for details);
    \item the symplectic form on $U$ is
      \[\omega = \int_{0}^{1} (\varphi_X^s)^*\omega_0 ds,\]
      where $\omega_0$ is the standard symplectic structure on $T^*M$.
  \end{enumerate}
\end{theorem}

\begin{remark}
  By a local symplectic groupoid $(\cG, \omega) \rra (M, \pi)$, we mean that the multiplication $m: \cG {_\alpha \times_\beta} \cG \to \cG$ may not be defined on all of its domain.
  In general, the local symplectic groupoid structure cannot be extended to the total space $T^*M$.
  Indeed, the Poisson spray $X$ may not be complete; the flow of the Poisson spray $X$ may contain loops; or the 2-form $\omega$, though non-degenerate near the zero section of $T^*M$, may be degenerate in general.
\end{remark}

\begin{remark}
  There are different sign conventions for the natural symplectic structure $\omega_0$ on $T^*M$. We follow the convention of \cite{MR2900786, CMS17}: $\omega_0 = - d \theta_0$ where $\theta_0$ is the tautological 1-form. In local coordinates $(\mathbf{x}, \mathbf{p})$, we have $\omega_0 = \sum_i dx_i \wedge dp_i$. This sign convention ensures that the source map $\alpha: (U, \omega) \to (M, \pi)$ is Poisson.
\end{remark}

In the next results, we introduce a Poisson spray whose local symplectic groupoid provides an integration of a log-canonical Poisson structure.
\begin{lemma}
  For the log-canonical Poisson structure $\{x_i, x_j\} = \Omega_{ij} x_i x_j$ on $\RR^m$, the vector field $X \in \cT_{T^*L}$ given by
  \begin{equation} 
    \label{eq: PoisSp}
    X = \sum_{1 \leq i,j \leq m}\Omega_{ij}x_i p_i x_j\frac{\partial}{\partial x_j} - \sum_{1 \leq i,j \leq m}\Omega_{ij}x_ip_i p_j\frac{\partial}{\partial p_j}
  \end{equation}
  is a Poisson spray.
  Its flow is given by
  \[\varphi_X^t: T^*L \to T^*L, \qquad (\bfp, \bfx) \mapsto (\bfa^{-t} \circ \bfp, \bfa^t \circ \bfx),\]
  where $a_j = e^{\sum_i \Omega_{ij} x_ip_i}$.
  This flow exists for all time $t \in \RR$ and contains no loops.
\end{lemma}

\begin{proof}
	For the co-vector $\alpha = p_1 dx_1 + \ldots + p_m dx_m$ and the point $(\mathbf{x}, \mathbf{p})$, we have
	\[
		\tau_* X|_{(\mathbf{x},\mathbf{p})} = \sum_{i, j}\Omega_{ij} x_i p_i x_j \frac{\partial}{\partial x_j} = \iota_\alpha \pi.
	\]
	To see the flow of $X$, we note that $x_i p_i$ is a constant under the flow of $X$, since
	\[
		\frac{d}{dt}(x_j p_j) = \dot{x}_j p_j - x_j \dot{p}_j = \sum_{1 \leq i,j \leq m}\Omega_{ij}x_i p_i x_j p_j - \sum_{1 \leq i,j \leq m}\Omega_{ij}x_ip_i p_j x_j = 0.
	\]
	Therefore, $\sum_{i}\Omega_{ij}x_i p_i x_j$ is constant, and
	\begin{align*}
		x_j(t) & = e^{t \sum_i \Omega_{ij} x_ip_i} x_j(0) = a_j^t x_j(0), \\
		p_j(t) & = e^{-t \sum_i \Omega_{ij} x_ip_i} p_j(0) = a_j^{-t} p_j(0).
	\end{align*}
\end{proof}

This Poisson spray $X$ induces the symplectic groupoid structure below.
\begin{theorem} 
  \label{thm:PoiSpLogC}
  For the log-canonical Poisson structure $\{x_i, x_j\} = \Omega_{ij} x_i x_j$ on $L$ (which is either $\RR^m$ or $\CC^m$), there is a source-simply-connected symplectic groupoid $(\cG_L, \omega) \rra (L, \pi)$ with the following structures:
  \begin{enumerate}
    \item $\cG_L \cong T^*L$ with coordinates $(\bfp, \bfx) = (p_1, \ldots, p_m, x_1, \ldots, x_m)$;
    \item the source map is the bundle projection $\alpha: T^*L \to L, \quad (\bfp, \bfx) \mapsto \bfx$;
    \item the target map is $\beta: T^*L \to L, \quad (\bfp, \bfx) \mapsto \bfa \circ \bfx$, where $a_j = e^{\sum_i \Omega_{ij} x_ip_i}$;
    \item the inverse map is $\iota: T^*L \to T^*L, \quad (\bfp, \bfx) \mapsto -\bfa^{-1}\circ \bfx$;
    \item the multiplication map is $m: T^*L {_\alpha \times_\beta} T^*L \to T^*L, \quad \big((\bfp, \bfx), (\bfp', \bfa \circ \bfx)\big) \mapsto (\bfa \circ \bfp' + \bfp, \bfx )$;
    \item the symplectic form $\omega$ is
      \begin{equation}
        \label{eq:2-formG_A}
	\omega = \sum_{i} dx_i \wedge dp_i
	  + \left(
	    \sum_{i, j} \Omega_{ij}p_ix_j dx_i \wedge dp_j 
	    + \sum_{j < i} \Omega_{ij}p_ip_j dx_i \wedge dx_j
	    + \sum_{j < i} \Omega_{ij}x_ix_j dp_i \wedge dp_j
	  \right).
      \end{equation}
  \end{enumerate}
\end{theorem}
\begin{proof}
  When the underlying field is $\RR$, it is straightforward to check that the Poisson spray \eqref{eq: PoisSp} induces the given source-simply-connected symplectic groupoid structures.
  Note that the multiplicative 2-form $\omega$ is non-degenerate since
  \[\omega^m = m! \bigwedge\limits_{1\leq i\leq m} dx_i \wedge dp_i\]
  is a volume form.

  In the case when the underlying field is $\CC$, the symplectic groupoid structures can be directly verified.
\end{proof}

For the next construction we restrict $L$ to be either $\RR_+^m$ or $(\CC^\times)^m$, where $\RR_+ = (0, \infty)$ and $\CC^\times = \CC \setminus \{0\}$.
Motiviated by \cite{MR2470108}, we make the following definition of the symplectic double, which is a source-connected symplectic groupoid of the log-canonical Poisson structure on $L$.
\begin{theorem} 
  \cite{MR2470108}
  For the log-canonical Poisson structure $\{x_i, x_j\} = \Omega_{ij} x_i x_j$ on $L$ (which is either $\RR_+^m$ or $(\CC^*)^m$), the \emph{symplectic double} $(\cD_L, \omega) \rra (L, \pi)$ is a source-connected symplectic groupoid with the following structures:
  \begin{enumerate}
    \item $\cG_L \cong L \times L$ with coordinates $(\bfs, \bfx) = (s_1, \ldots, s_m, x_1, \ldots, x_m)$;
    \item the source map is
      \[\alpha: L \times L \to L, \qquad (\bfs, \bfx) \mapsto \bfx;\]
    \item the target map is
      \[\beta: L \times L \to L, \qquad (\bfs, \bfx) \mapsto \left(x_1 \prod_{i=1}^m s_i^{\Omega_{i1}}, \ldots, x_m \prod_{i=1}^m s_i^{\Omega_{im}}\right);\]
    \item the inverse map is
      \[\iota: L \times L \to L \times L, \qquad (\bfs, \bfx) \mapsto \left(\frac{1}{s_1}, \ldots, \frac{1}{s_m}, x_1 \prod_{i=1}^m s_i^{\Omega_{i1}}, \ldots, x_m \prod_{i=1}^m s_i^{\Omega_{im}}\right);\]
    \item the multiplication map is
      \[\begin{aligned}
	  m: & \left(L \times L\right) {_\alpha \times_\beta} \left(L \times L\right) \to L \times L, \\
	     & \left((\bfs, \bfx), \left(\bfs', x_1 \prod_{i=1}^m s_i^{\Omega_{i1}}, \ldots, x_m \prod_{i=1}^m s_i^{\Omega_{im}}\right)\right) \mapsto (\bfs' \circ \bfs, \bfx );
        \end{aligned}\]
    \item the symplectic form $\omega$ is
      \begin{equation} 
        \label{eq:2-formG_X}
	\omega = \sum_{i} \frac{dx_i}{x_i} \wedge \frac{ds_i}{s_i} + \sum_{j < i} \Omega_{ij} \frac{ds_i}{s_i} \wedge \frac{ds_j}{s_j}.
      \end{equation}
  \end{enumerate}
\end{theorem}

\footnote{double-check the groupoid structures. Also expend the remark on the notations. Compare it to the notations in \cite{MR2470108}.}

Similar to the case of Lie groups, for an integrable Lie algebroid, every source-connected integration receives a surjective groupoid map from the source-simply-connected groupoid.
With our chosen conventions, we have a simple expression for the map $\kappa: \cG_L \to \cD_L$.
\begin{proposition}
  The covering map
  $$\kappa: \cG_L \to \cD_L, \qquad (\bfp, \bfx) \mapsto (e^{x_1p_1}, \ldots, e^{x_mp_m}, \bfx)$$
  is a groupoid morphism.
\end{proposition}


%%%%%%%%%%%%%%%%%%%%%%%%%%%%%%%%%%%%%%%%%%%%%%
\section{Hamiltonian Perspective on Mutations}
\label{sec:cluster}

In this section we introduce mutations of cluster seeds from the Hamiltonian viewpoint.
This perspective was first given in \cite{MR2470108} and is the foundation for the main results of \cite{MR3691969}.
We demonstrate how the Hamiltonian perspective provides a canonical choice of mutations for groupoid charts which glue to give a symplectic groupoid integrating various log-canonical Poisson structure on cluster varieties.

Let $\tilde B=(B_{ij})$ be an $m\times n$ integer matrix with $m\ge n$.  
Write $B$ for the upper $n\times n$ submatrix of $\tilde B$ and assume $B$ is skew-symmetrizable, i.e. there exists a diagonal integer matrix $D=\diag(d_1,\ldots,d_n)$ with each $d_i>0$ so that $DB$ is skew-symmetric. 
Such an $m\times n$ matrix $\tilde B$ with skew-symmetrizable principal submatrix $B$ is called an \emph{exchange matrix}.
We fix a skew-symmetrizing matrix $D$.
An $m\times m$ matrix $\Omega=(\Omega_{ij})$ is \emph{$D$-compatible} with $\tilde B$ if $\tilde B^T\Omega=[D\ \boldsymbol{0}]$, where $\boldsymbol{0}$ denotes an $n\times(m-n)$ matrix with all zero entries.
In this case, we call $(\tilde B,\Omega)$ a \emph{$D$-compatible pair}.

\subsection{Mutation of Cluster Charts}
Let $L_\cX=\RR_+^n$ and $L_\cA=\RR_+^m$.
Write $\bfy=(y_1,\ldots,y_n)$ for a set of coordinates on $L_\cX$ and $\bfx=(x_1,\ldots,x_m)$ for a set of coordinates on $L_\cA$.
Given a $D$-compatible pair of matrices $(\tilde B,\Omega)$, denote by $\{\cdot,\cdot\}_\cX$ and $\{\cdot,\cdot\}_\cA$ the log-canonical Poisson brackets on $L_\cX$ and $L_\cA$ given by
\begin{equation}
  \label{eq:brackets}
  \{y_k,y_\ell\}_\cX=d_kB_{k\ell}y_ky_\ell\qquad\text{and}\qquad\{x_i,x_j\}_\cA=\Omega_{ij}x_ix_j.
\end{equation}
An easy calculation shows that the map
\[\rho:L_\cA\to L_\cX,\qquad \bfx\mapsto\hat\bfy,\qquad \hat y_k:=\prod_{i=1}^m x_i^{B_{ik}}\]
is a Poisson morphism, i.e.\ $\{\hat y_k,\hat y_\ell\}_\cA=d_kB_{k\ell}\hat y_k\hat y_\ell$ for $1\le k,\ell\le n$.
Motivated by the terminology of Fock and Goncharov \cite{???}, we will refer to the pair of Poisson manifolds $L_\cX$ and $L_\cA$ together with the Poisson morphism $\rho$ as a \emph{Poisson ensemble} associated to the $D$-compatible pair $(\tilde B,\Omega)$.

The \emph{Euler dilogarithm} is the function of a single real variable defined by
\[\Li_2(y)=-\int_0^y \frac{\log(1-u)}{u}du,\qquad y<1.\]
We will see below that, from the Poisson perspective, the Euler dilogarithm lies at the heart of cluster algebra theory.

Given a choice of sign $\varepsilon\in\{\pm1\}$, define Hamiltonian functions $H_\cX=H^{k,\varepsilon}_\cX\in\cO_{L_\cX}$ and $H_\cA=H^{k,\varepsilon}_\cA\in\cO_{L_\cA}$ for $1\le k\le n$ by
\begin{equation}
  \label{eq:hamiltonians}
  H_\cX^{k,\varepsilon}:=\frac{\varepsilon}{d_k}\Li_2(-y_k^\varepsilon)\qquad\text{and}\qquad H_\cA^{k,\varepsilon}:=\frac{\varepsilon}{d_k}\Li_2(-\hat y_k^\varepsilon).
\end{equation}
Clearly we have $\rho^*(H_\cX^{k,\varepsilon})=H_\cA^{k,\varepsilon}$.
Write $X_\cX=X_\cX^{k,\varepsilon}\in\cT_\cX$ and $X_\cA=X_\cA^{k,\varepsilon}\in\cT_\cA$ for the Hamiltonian vector fields associated to $H_\cX^{k,\varepsilon}$ and $H_\cA^{k,\varepsilon}$ respectively, i.e.\ the vector fields naturally associated to the derivations $\{H_\cX^{k,\varepsilon},\cdot\}_\cX$ and $\{H_\cA^{k,\varepsilon},\cdot\}_\cA$.
\begin{lemma}
  \label{le:hamiltonian dynamics}
  For $1\le k\le n$ and $\varepsilon\in\{\pm1\}$, the vector fields $X_\cX$ and $X_\cA$ determine the following dynamics on $L_\cX$ and $L_\cA$:
  \begin{align}
    \label{eq:X dynamics}
    \dot y_\ell&:=\{H_\cX^{k,\varepsilon},y_\ell\}_\cX=-B_{k\ell}\log(1+y_k^\varepsilon)y_\ell\quad\text{for}\quad 1\le\ell\le n;\\
    \label{eq:A dynamics}
    \dot x_j&:=\{H_\cA^{k,\varepsilon},x_j\}_\cA=-\delta_{jk}\log(1+\hat y_k^\varepsilon)x_k\quad\text{for}\quad 1\le j\le m.
  \end{align}
\end{lemma}
\begin{proof}
  Equation~\eqref{eq:X dynamics} was essentially proven in \cite{MR3691969}.
  Since equation~\eqref{eq:A dynamics} seems to be new, we will indicate the key steps in the computation here:
  \[\{H_\cA^{k,\varepsilon},x_j\}_\cA=-\frac{\log(1+\hat y_k^\varepsilon)}{d_k\hat y_k}\{\hat y_k,x_j\}=-\delta_{jk}\log(1+\hat y_k^\varepsilon)x_j.\]
\end{proof}

It immediately follows that $y_k$ is a conserved quantity of the flow of $L_\cX$ by the vector field $X_\cX$.
Since the map $\rho$ is Poisson with $\hat y_k=\rho^*(y_k)$, we also see that $\hat y_k$ is a conserved quantity of the flow of $L_\cA$ by the vector field $X_\cA$.
Write $\varphi_\cX^1:L_\cX\to L_\cX$ for the time-one flow of the vector field $X_\cX^{k,\varepsilon}$.
Similarly, we denote by $\varphi_{X_\cA}^1:L_\cA\to L_\cA$ the time-one flow of the vector field $X_\cA^{k,\varepsilon}$.
\begin{corollary}
  \label{cor:time-one flows}
  For $1\le k\le n$ and $\varepsilon\in\{\pm1\}$, we have
  \begin{align*}
    (\varphi_\cX^1)^*(y_\ell)&=(1+y_k^\varepsilon)^{-B_{k\ell}}y_\ell;\\
    (\varphi_\cA^1)^*(x_j)&=(1+\hat y_k^\varepsilon)^{-\delta_{jk}}x_j.
  \end{align*}
\end{corollary}
\begin{proof}
  This is immediate from the preceding discussion.
\end{proof}

In what follows we use the notation $[a]_+=\max\{a,0\}$.
For $1\le k\le n$ and a sign $\varepsilon\in\{\pm1\}$, define the \emph{mutation of the pair $(\tilde B,\Omega)$ in direction $k$} by $\mu_{k,\varepsilon}(\tilde B,\Omega)=\big((B'_{ij}),\Omega'\big)$, where
\begin{itemize}
  \item $\mu_{k,\varepsilon}\tilde B=(B'_{ij})$ is given by
    \[B'_{ij}=\begin{cases}-B_{ij} & \text{if $i=k$ or $j=k$;}\\ B_{ij}+[-\varepsilon B_{ik}]_+B_{kj}+B_{ik}[\varepsilon B_{kj}]_+ & \text{otherwise;}\end{cases}\]
  \item $\Omega'=E_{k,\varepsilon}^T\Omega E_{k,\varepsilon}$ for $E_{k,\varepsilon}$ the $m\times m$ matrix with entries
    \[E_{ij}=\begin{cases}\delta_{ij} & \text{if $j\ne k$;}\\ -1 & \text{if $i=j=k$;}\\ [-\varepsilon B_{ik}]_+ & \text{if $i\ne j=k$.}\end{cases}\]
\end{itemize}
It is an easy exercise to check that matrix mutation is an involution, more precisely $\mu_{k,\varepsilon}\mu_{k,\varepsilon'}(\tilde B,\Omega)=(\tilde B,\Omega)$ for any signs $\varepsilon,\varepsilon'\in\{\pm1\}$.
In particular, the mutation $\mu_{k,\varepsilon}$ acting on $D$-compatible pairs is independent of the sign $\varepsilon$ and we simply write $\mu_k$ for this mutation when the sign does not matter.

Let $L'_\cX=\RR_+^n$ and $L'_\cA=\RR_+^m$ with coordinates $\bfy'=(y'_1,\ldots,y'_n)$ and $\bfx'=(x'_1,\ldots,x'_m)$ respectively.
For $1\le k\le n$ and a sign $\varepsilon\in\{\pm1\}$, set $(\tilde B',\Omega')=\mu_{k,\varepsilon}(\tilde B,\Omega)$ and consider the log-canonical Poisson structures $\{\cdot,\cdot\}'_\cX$ and $\{\cdot,\cdot\}'_\cA$ on $L'_\cX$ and $L'_\cA$ given by
\begin{equation}
  \label{eq:brackets}
  \{y'_k,y'_\ell\}'_\cX=d_kB'_{k\ell}y'_ky'_\ell\qquad\text{and}\qquad\{x'_i,x'_j\}'_\cA=\Omega'_{ij}x'_ix'_j.
\end{equation}
\begin{lemma}
  \label{le:tropical cluster transformations}
  For $1\le k\le n$ and $\varepsilon\in\{\pm1\}$, there are Poisson morphisms $\tau_\cX^{k,\varepsilon}:L_\cX\to L'_\cX$ and $\tau_\cA^{k,\varepsilon}:L_\cA\to L'_\cA$ given on coordinates by
  \begin{align}
    \label{eq:tropical X transformation}
    (\tau_\cX^{k,\varepsilon})^*(y'_\ell)
    &=\begin{cases} 
      y_k^{-1} & \text{if $\ell=k$;}\\ 
      y_\ell y_k^{[\varepsilon B_{k\ell}]_+} & \text{if $\ell\ne k$;}
    \end{cases}\\
    \label{eq:tropical A transformation}
    (\tau_\cA^{k,\varepsilon})^*(x'_j)
    &=\begin{cases} 
      x_k^{-1}\prod\limits_{i=1}^m x_i^{[-\varepsilon B_{ik}]_+} & \text{if $j=k$;}\\
      x_j & \text{if $j\ne k$.}
    \end{cases}
  \end{align}
\end{lemma}
\begin{proof}
  By skew-symmetry of the Poisson brackets, there are essentially only two cases to check and just one of these is non-trivial for each transformation.
  For $\ell,\ell'\ne k$, we have
  \begin{align*}
    \{(\tau_\cX^{k,\varepsilon})^*(y'_\ell),(\tau_\cX^{k,\varepsilon})^*(y'_{\ell'})\}_\cX
    &=\left\{y_\ell y_k^{[\varepsilon B_{k\ell}]_+},y_{\ell'} y_k^{[\varepsilon B_{k\ell'}]_+}\right\}_\cX\\
    &=(d_\ell B_{\ell\ell'}+d_k[\varepsilon B_{k\ell}]_+B_{k\ell'}+d_\ell B_{\ell k}[\varepsilon B_{k\ell'}]_+) y_\ell y_k^{[\varepsilon B_{k\ell}]_+} y_{\ell'} y_k^{[\varepsilon B_{k\ell'}]_+}\\
    &=d_\ell B'_{\ell\ell'} (\tau_\cX^{k,\varepsilon})^*(y'_\ell) (\tau_\cX^{k,\varepsilon})^*(y'_{\ell'}),
  \end{align*}
  where the last equality uses the identity $d_kB_{k\ell}=-d_\ell B_{\ell k}$.
  The case where $\ell$ or $\ell'$ are equal to $k$ is immediate and we omit the details.

  For $\tau_\cA^{k,\varepsilon}$, we only check the brackets of $x'_k$ and $x'_j$ for $j\ne k$:
  \begin{align*}
    \{(\tau_\cA^{k,\varepsilon})^*(x'_k),(\tau_\cA^{k,\varepsilon})^*(x'_j)\}_\cA
    &=\left\{x_k^{-1}\prod\limits_{i=1}^m x_i^{[-\varepsilon B_{ik}]_+},x_j\right\}_\cA\\
    &=\left(-\Omega_{kj}+\sum_{i=1}^m [-\varepsilon B_{ik}]_+\Omega_{ij}\right) \left(x_k^{-1} \prod\limits_{i=1}^m x_i^{[-\varepsilon B_{ik}]_+}\right)x_j\\
    &=\Omega'_{kj} (\tau_\cA^{k,\varepsilon})^*(x'_k) (\tau_\cA^{k,\varepsilon})^*(x'_j).
  \end{align*}
\end{proof}

For $1\le k\le n$ and $\varepsilon\in\{\pm1\}$, define the \emph{cluster mutations in direction $k$} by $\mu_\cX^{k,\varepsilon}:=\tau_\cX^{k,\varepsilon}\circ\varphi_\cX^1:L_\cX\to L'_\cX$ and $\mu_\cA^{k,\varepsilon}:=\tau_\cA^{k,\varepsilon}\circ\varphi_\cA^1:L_\cA\to L'_\cA$.
\begin{lemma}
  \label{le:cluster mutation}
  For $1\le k\le n$ and $\varepsilon\in\{\pm1\}$, the cluster mutations provide Poisson morphisms $\mu_\cX^{k,\varepsilon}:L_\cX\to L'_\cX$ and $\mu_\cA^{k,\varepsilon}:L_\cA\to L'_\cA$ given on coordinates by
  \begin{align}
    \label{eq:X mutation}
    (\mu_\cX^{k,\varepsilon})^*(y'_\ell)&=\begin{cases} y_k^{-1} & \text{if $\ell=k$;}\\ y_\ell y_k^{[\varepsilon B_{k\ell}]_+}(1+y_k^\varepsilon)^{-B_{k\ell}} & \text{if $\ell\ne k$;}\end{cases}\\
    \label{eq:A mutation}
    (\mu_\cA^{k,\varepsilon})^*(x'_j)&=\begin{cases} x_k^{-1}\left(\prod\limits_{i=1}^m x_i^{[-\varepsilon B_{ik}]_+}\right)(1+\hat y_k^\varepsilon) & \text{if $j=k$;}\\ x_j & \text{if $j\ne k$.}\end{cases}
  \end{align}
\end{lemma}
\begin{proof}
  Observe for both $L_\cX$ and $L_\cA$ that $(\mu_-^{k,\varepsilon})^*=(\varphi_-^{k,\varepsilon})^*\circ(\tau_-^{k,\varepsilon})^*$ and so the result immediately follows by combining Corollary~\ref{cor:time-one flows} and Lemma~\ref{le:tropical cluster transformations}.
\end{proof}

\subsection{Mutation of Groupoid Charts}
We have presented thus far the Hamiltonian viewpoint of mutation for cluster charts.
Our goal now is to apply the groupoid formalism to lift these results to the level of the symplectic groupoids $\cG_\cX$, $\cD_\cX$ and $\cG_\cA$, $\cD_\cA$ integrating the standard Poisson structures on $L_\cX$ and $L_\cA$ respectively.
These groupoids have coordinates given as follows:
\begin{itemize}
  \item $\cG_\cX$ has coordinates $(\bfq,\bfy)=(q_1,\ldots,q_n,y_1,\ldots,y_n)$; 
  \item $\cD_\cX$ has coordinates $(\bft,\bfy)=(t_1,\ldots,t_n,y_1,\ldots,y_n)$; 
  \item $\cG_\cA$ has coordinates $(\bfp,\bfx)=(p_1,\ldots,p_m,x_1,\ldots,x_m)$; 
  \item $\cD_\cA$ has coordinates $(\bfs,\bfx)=(s_1,\ldots,s_m,x_1,\ldots,x_m)$.
\end{itemize}
For simplicity of notation, we will write $\alpha$ and $\beta$ for the source and target maps of all the groupoids $\cG_\cX$, $\cD_\cX$, $\cG_\cA$, and $\cD_\cA$; this slight abuse of notation should not lead to any confusion.
Write $\kappa_\cX:\cG_\cX\to\cD_\cX$ and $\kappa_\cA:\cG_\cA\to\cD_\cA$ for the groupoid covering maps from the source-simply connected symplectic groupoids to the symplectic doubles, these are given on fiber coordinates by
\[\kappa_\cX^*(t_\ell)=e^{q_\ell y_\ell}\quad\text{and}\quad\kappa_\cA^*(s_j)=e^{p_j x_j}.\]
\begin{lemma}
  \footnote{Check this again}
  The Poisson ensemble map $\rho:L_\cA\to L_\cX$ lifts to a morphism of symplectic groupoids $\tilde\rho:\cG_\cA\to\cG_\cX$ given on fiber coordinates by 
  \[\tilde\rho^*(q_\ell)=-(d_\ell\hat y_\ell)^{-1}\sum\limits_{j=1}^m\Omega_{j\ell}x_jp_j.\]
\end{lemma}
\begin{remark}
  When $L_\cX=\RR_+^n$ and $L_\cA=\RR_+^m$, the Poisson ensemble map $\rho:L_\cA\to L_\cX$ also lifts to a morphism of symplectic groupoids $\bar\rho:\cD_\cA\to\cD_\cX$ given on fiber coordinates by 
  \footnote{This might actually require $m=n$.}
  \[\bar\rho^*(t_\ell)=\prod\limits_{j=1}^m s_j^{-d_\ell^{-1}\Omega_{\ell j}}.\]
\end{remark}

A crucial observation will be the following explicit computation of the Poisson structures on each of these symplectic groupoids.
\begin{lemma}
  \label{le:groupoid Poisson structures}
  \begin{align}
    \pi_{\cG_\cX}:=\omega_{\cG_\cX}^{-1}&=\sum_{k} \frac{\partial}{\partial q_k} \wedge \frac{\partial}{\partial y_k} - \sum_{k, \ell} d_kB_{k\ell}q_k y_\ell \frac{\partial}{\partial q_k} \wedge \frac{\partial}{\partial y_\ell}\\
    \nonumber &\quad+\sum_{\ell < k} d_kB_{k\ell}q_kq_\ell \frac{\partial}{\partial q_k} \wedge \frac{\partial}{\partial q_\ell}
		    +\sum_{\ell < k} d_kB_{k\ell}y_ky_\ell \frac{\partial}{\partial y_k} \wedge \frac{\partial}{\partial y_\ell};\\
    \pi_{\cD_\cX}:=\omega_{\cD_\cX}^{-1}&=\sum_{k} t_ky_k\frac{\partial}{\partial t_k} \wedge \frac{\partial}{\partial y_k}+\sum_{\ell < k} d_kB_{k\ell}y_ky_\ell \frac{\partial}{\partial y_k} \wedge \frac{\partial}{\partial y_\ell};\\
    \pi_{\cG_\cA}:=\omega_{\cG_\cA}^{-1}&=\sum_{i} \frac{\partial}{\partial p_i} \wedge \frac{\partial}{\partial x_i} - \sum_{i, j} \Omega_{ij}p_i x_j \frac{\partial}{\partial p_i} \wedge \frac{\partial}{\partial x_j}\\
    \nonumber &\quad+\sum_{j < i} \Omega_{ij}p_ip_j \frac{\partial}{\partial p_i} \wedge \frac{\partial}{\partial p_j}
		    +\sum_{j < i} \Omega_{ij}x_ix_j \frac{\partial}{\partial x_i} \wedge \frac{\partial}{\partial x_j};\\
    \pi_{\cD_\cA}:=\omega_{\cD_\cA}^{-1}&=\sum_{i} s_ix_i\frac{\partial}{\partial s_i} \wedge \frac{\partial}{\partial x_i}+\sum_{j < i} \Omega_{ij}x_ix_j \frac{\partial}{\partial x_i} \wedge \frac{\partial}{\partial x_j}.
  \end{align}
\end{lemma}
\begin{proof}
  We prove these statements only for the groupoids over $L_\cA$, the corresponding statements over $L_\cX$ follow by a simple change of notation.

  Choosing the standard frames: $\{dx_i, dp_i\}_i$ for $T^*\cG_\cA$ and $\left\{\frac{\partial}{\partial x_i}, \frac{\partial}{\partial p_i}\right\}_i$ for $T\cG_\cA$, the bundle map $\pi_{\cG_\cA}^\flat: T^*\cG_\cA \to T\cG_\cA, \quad \alpha \mapsto \pi_{\cG_\cA}(\alpha, \cdot)$ is given by
  \begin{align*}
    \pi_{\cG_\cA}^\flat: &~ dx_i \mapsto -\frac{\partial}{\partial p_i} - \sum_{j} \Omega_{ij} x_i p_j \frac{\partial}{\partial p_j}+ \sum_{j} \Omega_{ij} x_i x_j \frac{\partial}{\partial x_j}, \\
    &~ dp_i \mapsto \frac{\partial}{\partial x_i} - \sum_{j} \Omega_{ij} p_i x_j \frac{\partial}{\partial x_j}+ \sum_{j} \Omega_{ij} p_i p_j \frac{\partial}{\partial p_j},
  \end{align*}
  and for the symplectic structure $\omega_{\cG_\cA}$ as in \eqref{eq:2-formG_A}, the bundle map $\omega_{\cG_\cA}^\sharp: T\cG_\cA \to T^*\cG_\cA, \quad v \mapsto \omega_{\cG_\cA}(v, \cdot)$ is given by
  \begin{align*}
    \omega_{\cG_\cA}^\sharp: &~ \frac{\partial}{\partial x_i} \mapsto dp_i + \sum_{j} \Omega_{ij} p_i x_j dp_j + \sum_{j} \Omega_{ij} p_i p_j dx_j, \\
    &~ \frac{\partial}{\partial p_i} \mapsto -dx_i + \sum_{j} \Omega_{ij} x_i p_j dx_j + \sum_{j} \Omega_{ij} x_i x_j dp_j.
  \end{align*}

  Similarly, choosing the standard frames: $\{dx_i, ds_i\}_i$ for $T^*\cD_\cA$ and $\left\{\frac{\partial}{\partial x_i}, \frac{\partial}{\partial s_i}\right\}_i$ for $T\cD_\cA$, the bundle map $\pi_{\cD_\cA}^\flat: T^*\cD_\cA \to T\cD_\cA, \quad \alpha \mapsto \pi_{\cD_\cA}(\alpha, \cdot)$ is given by
  \begin{align*}
    \pi_{\cD_\cA}^\flat: &~ dx_i \mapsto -x_i s_i \frac{\partial}{\partial s_i} + \sum_{j} \Omega_{ij} x_i x_j \frac{\partial}{\partial x_j}, \\
    &~ ds_i \mapsto s_i x_i \frac{\partial}{\partial x_i},
  \end{align*}
  and for the symplectic structure $\omega_{\cD_\cA}$ as in \eqref{eq:2-formG_X}, the bundle map $\omega_{\cD_\cA}^\sharp: T\cD_\cA \to T^*\cD_\cA, \quad v \mapsto \omega_{\cD_\cA}(v, \cdot)$ is given by
  \begin{align*}
    \omega_{\cD_\cA}^\sharp: &~ \frac{\partial}{\partial x_i} \mapsto \frac{1}{x_i s_i} ds_i, \\
    &~ \frac{\partial}{\partial s_i} \mapsto -\frac{1}{x_i s_i} dx_i + \sum_{j} \frac{\Omega_{ij}}{s_i s_j} ds_j.
  \end{align*}

  We leave it to the reader to check that $\pi_{\cG_\cA}^\flat$ (resp. $\pi_{\cD_\cA}^\flat$) and $\omega_{\cG_\cA}^\sharp$ (resp. $\omega_{\cD_\cA}^\sharp$) are inverse to each other.
\end{proof}

Define functions $\tilde H_\cX^{k,\varepsilon}\in\cO_{\cG_\cX}$ and $\bar H_\cX^{k,\varepsilon}\in\cO_{\cD_\cX}$ by the formulas $\alpha^* H_\cX^{k,\varepsilon}-\beta^* H_\cX^{k,\varepsilon}$ for appropriate source and target maps. \footnote{Elaborate on these being multiplicative.}
Define functions $\tilde H_\cA^{k,\varepsilon}\in\cO_{\cG_\cA}$ and $\bar H_\cA^{k,\varepsilon}\in\cO_{\cD_\cA}$ similarly.
Since $\kappa_\cX:\cG_\cX\to\cD_\cX$ and $\kappa_\cA:\cG_\cA\to\cD_\cA$ are groupoid morphisms, we clearly have $\kappa_\cX^*(\bar H_\cX^{k,\varepsilon})=\tilde H_\cX^{k,\varepsilon}$ and $\kappa_\cA^*(\bar H_\cA^{k,\varepsilon})=\tilde H_\cA^{k,\varepsilon}$.
More explicitly, these Hamiltonian functions are given by
\begin{align*}
  \tilde H_\cX^{k,\varepsilon}&=\frac{\varepsilon}{d_k}\Li_2(-y_k^\varepsilon)-\frac{\varepsilon}{d_k}\Li_2\left(-y_k^\varepsilon e^{\varepsilon\sum_{\ell=1}^n d_\ell B_{\ell k}q_\ell y_\ell}\right);\\
  \bar H_\cX^{k,\varepsilon}&=\frac{\varepsilon}{d_k}\Li_2(-y_k^\varepsilon)-\frac{\varepsilon}{d_k}\Li_2\left(-y_k^\varepsilon\prod_{\ell=1}^n t_\ell^{\varepsilon d_\ell B_{\ell k}}\right);\\
  \tilde H_\cA^{k,\varepsilon}&=\frac{\varepsilon}{d_k}\Li_2(-\hat y_k^\varepsilon)-\frac{\varepsilon}{d_k}\Li_2\left(-\hat y_k^\varepsilon e^{-\varepsilon d_kp_kx_k}\right);\\
  \bar H_\cA^{k,\varepsilon}&=\frac{\varepsilon}{d_k}\Li_2(-\hat y_k^\varepsilon)-\frac{\varepsilon}{d_k}\Li_2\left(-\hat y_k^\varepsilon s_k^{-\varepsilon d_k}\right).
\end{align*}
Write $\tilde X_\cX^{k,\varepsilon}\in\cT_{\cG_\cX}$, $\bar X_\cX^{k,\varepsilon}\in\cT_{\cD_\cX}$, $\tilde X_\cA^{k,\varepsilon}\in\cT_{\cG_\cA}$, and $\bar X_\cA^{k,\varepsilon}\in\cT_{\cD_\cA}$ for the Hamiltonian vector fields associated to the functions $\tilde H_\cX^{k,\varepsilon}$, $\bar H_\cX^{k,\varepsilon}$, $\tilde H_\cA^{k,\varepsilon}$, and $\bar H_\cA^{k,\varepsilon}$ respectively, i.e.\ they are the vector fields naturally associated to the derivations $\{\tilde H_\cX^{k,\varepsilon},\cdot\}_{\cG_\cX}$, $\{\bar H_\cX^{k,\varepsilon},\cdot\}_{\cD_\cX}$, $\{\tilde H_\cA^{k,\varepsilon},\cdot\}_{\cG_\cA}$, and $\{\bar H_\cA^{k,\varepsilon},\cdot\}_{\cD_\cA}$.
\begin{lemma}
  For $1\le k\le n$ and $\varepsilon\in\{\pm1\}$, the vector fields $\tilde X_\cX^{k,\varepsilon}\in\cT_{\cG_\cX}$, $\bar X_\cX^{k,\varepsilon}\in\cT_{\cD_\cX}$, $\tilde X_\cA^{k,\varepsilon}\in\cT_{\cG_\cA}$, and $\bar X_\cA^{k,\varepsilon}\in\cT_{\cD_\cA}$ determine the following dynamics on the symplectic groupoids $\cG_\cX$, $\cD_\cX$, $\cG_\cA$, and $\cD_\cA$:
  \begin{align}
    \dot q_\ell&=\{\tilde H_\cX^{k,\varepsilon},q_\ell\}_{\cG_\cX}=-\frac{\delta_{k\ell}}{d_ky_k}\log\left(\frac{1+y_k^\varepsilon e^{\varepsilon\sum_{\ell'=1}^n d_{\ell'} B_{\ell' k}q_{\ell'} y_{\ell'}}}{1+y_k^\varepsilon}\right)+B_{k\ell}\log(1+y_k^\varepsilon)q_\ell;\\
    \dot t_\ell&=\{\bar H_\cX^{k,\varepsilon},t_\ell\}_{\cD_\cX}=-\frac{\delta_{k\ell}}{d_k}\log\left(\frac{1+y_k^\varepsilon \prod_{\ell'=1}^nt_{\ell'}^{\varepsilon d_{\ell'} B_{\ell' k}}}{1+y_k^\varepsilon}\right)t_k;\\
    \label{eq:p dot}\dot p_j&=\{\tilde H_\cA^{k,\varepsilon},p_j\}_{\cG_\cA}=-\frac{B_{jk}}{d_kx_j}\log\left(\frac{1+\hat y_k^\varepsilon e^{-\varepsilon d_kp_kx_k}}{1+\hat y_k^\varepsilon}\right)+\delta_{jk}\log(1+\hat y_k^\varepsilon)p_k;\\
    \label{eq:s dot}\dot s_j&=\{\bar H_\cA^{k,\varepsilon},s_j\}_{\cD_\cA}=-\frac{B_{jk}}{d_k}\log\left(\frac{1+\hat y_k^\varepsilon s_k^{-\varepsilon d_k}}{1+\hat y_k^\varepsilon}\right)s_j.
  \end{align}
\end{lemma}
\begin{proof}
  We prove equation \eqref{eq:p dot}, leaving the others as an exercise for the reader.
  By Lemma~\ref{le:groupoid Poisson structures}, we have $\{x_i,p_j\}_{\cG_\cA}=-\delta_{ij}-\Omega_{ij}x_ip_j$ and $\{p_i,p_j\}_{\cG_\cA}=\Omega_{ij}p_ip_j$ so that
  \[\{\hat y_k,p_j\}_{\cG_\cA}=\sum_{i=1}^m B_{ik}\frac{\hat y_k}{x_i}\{x_i,p_j\}_{\cG_\cA}=-B_{jk}\frac{\hat y_k}{x_j}-\delta_{jk}d_k\hat y_kp_k\]
  and
  \[\{e^{-d_kp_kx_k},p_j\}_{\cG_\cA}=\delta_{jk}d_kp_ke^{d_kp_kx_k}.\]
  Now equation~\eqref{eq:p dot} follows by applying the chain rule for the derivation $\{\cdot,p_j\}_{\cG_\cA}$.
\end{proof}

Consider the time-one flows of the vector fields $\tilde X_\cX^{k,\varepsilon}\in\cT_{\cG_\cX}$, $\bar X_\cX^{k,\varepsilon}\in\cT_{\cD_\cX}$, $\tilde X_\cA^{k,\varepsilon}\in\cT_{\cG_\cA}$, and $\bar X_\cA^{k,\varepsilon}\in\cT_{\cD_\cA}$:
\[
  \tilde\varphi_\cX^1:\cG_\cX\to\cG_\cX;\qquad
  \bar\varphi_\cX^1:\cD_\cX\to\cD_\cX;\qquad
  \tilde\varphi_\cA^1:\cG_\cA\to\cG_\cA;\qquad
  \bar\varphi_\cA^1:\cD_\cA\to\cD_\cA.
\]
\begin{corollary}
  For $1\le k\le n$ and $\varepsilon\in\{\pm1\}$, we have
  \begin{align*}
    (\tilde\varphi_\cX^1)^*(q_\ell y_\ell)&=q_\ell y_\ell-\frac{\delta_{k\ell}}{d_k}\log\left(\frac{1+y_k^\varepsilon e^{\varepsilon\sum_{\ell'=1}^n d_{\ell'} B_{\ell' k}q_{\ell'} y_{\ell'}}}{1+y_k^\varepsilon}\right);\\
    (\bar\varphi_\cX^1)^*(t_\ell)&=\left(\frac{1+y_k^\varepsilon \prod_{\ell'=1}^n t_{\ell'}^{\varepsilon d_{\ell'} B_{\ell' k}}}{1+y_k^\varepsilon}\right)^{-\delta_{k\ell}/d_k}t_\ell;\\
    (\tilde\varphi_\cA^1)^*(p_jx_j)&=p_jx_j-\frac{B_{jk}}{d_k}\log\left(\frac{1+\hat y_k^\varepsilon e^{-\varepsilon d_kp_kx_k}}{1+\hat y_k^\varepsilon}\right);\\
    (\bar\varphi_\cA^1)^*(s_j)&=\left(\frac{1+\hat y_k^\varepsilon s_k^{-\varepsilon d_k}}{1+\hat y_k^\varepsilon}\right)^{-B_{jk}/d_k}s_j.\\
  \end{align*}
\end{corollary}

\begin{lemma}
  \label{le:tropical groupoid transformations}
  For $1\le k\le n$ and $\varepsilon\in\{\pm1\}$, there are Poisson morphisms $\tilde\tau_\cX^{k,\varepsilon}:\cG_\cX\to\cG'_\cX$ and $\tilde\tau_\cA^{k,\varepsilon}:\cG_\cA\to\cG'_\cA$, lifting the Poisson morphisms of Lemma~\ref{le:tropical cluster transformations}, given on coordinates by
  \begin{align}
    \label{eq:tropical X transformation}
    (\tilde\tau_\cX^{k,\varepsilon})^*(q'_\ell)
    &=\begin{cases} 
      -q_k y_k^2 + \sum\limits_{\ell'=1}^n[\varepsilon B_{k\ell'}]_+ q_{\ell'} y_{\ell'} y_k & \text{if $\ell=k$;}\\ 
      q_\ell y_k^{-[\varepsilon B_{k\ell}]_+} & \text{if $\ell\ne k$;}
    \end{cases}\\
    \label{eq:tropical A transformation}
    (\tilde\tau_\cA^{k,\varepsilon})^*(p'_j)&=
    \begin{cases} 
      -p_k x_k^2 \prod\limits_{i=1}^m x_i^{-[-\varepsilon B_{ik}]_+} & \text{if $j=k$;}\\ 
      p_j+[-\varepsilon B_{jk}]_+\frac{p_k x_k}{x_j} & \text{if $j\ne k$.}
    \end{cases}
  \end{align}
\end{lemma}
\begin{proof}
  There are several cases to consider.
  For $\ell,\ell'\ne k$, we have
  \begin{align*}
    \{(\tilde\tau_\cX^{k,\varepsilon})^*(q'_\ell),(\tilde\tau_\cX^{k,\varepsilon})^*(q'_{\ell'})\}_{\cG_\cX}
    &=\left\{q_\ell y_k^{-[\varepsilon B_{k\ell}]_+},q_{\ell'} y_k^{-[\varepsilon B_{k\ell'}]_+}\right\}_{\cG_\cX}\\
    &=(d_\ell B_{\ell\ell'} + d_\ell B_{\ell k} [\varepsilon B_{k\ell'}]_+ + [\varepsilon B_{k\ell}]_+ d_k B_{k\ell'}) q_\ell y_k^{-[\varepsilon B_{k\ell}]_+} q_{\ell'} y_k^{-[\varepsilon B_{k\ell'}]_+}\\
    &=(d_\ell B_{\ell\ell'} + d_\ell B_{\ell k} [\varepsilon B_{k\ell'}]_+ + d_\ell [-\varepsilon B_{\ell k}]_+ B_{k\ell'}) q_\ell y_k^{-[\varepsilon B_{k\ell}]_+} q_{\ell'} y_k^{-[\varepsilon B_{k\ell'}]_+}\\
    &=d_\ell B'_{\ell\ell'} (\tilde\tau_\cX^{k,\varepsilon})^*(q'_\ell) (\tilde\tau_\cX^{k,\varepsilon})^*(q'_{\ell'})
  \end{align*}
  and
  \begin{align*}
    \{(\tilde\tau_\cX^{k,\varepsilon})^*(q'_\ell),(\tilde\tau_\cX^{k,\varepsilon})^*(y'_{\ell'})\}_{\cG_\cX}
    &=\left\{q_\ell y_k^{-[\varepsilon B_{k\ell}]_+},y_{\ell'} y_k^{[\varepsilon B_{k\ell'}]_+}\right\}_{\cG_\cX}\\
    &=\delta_{\ell\ell'} + (-d_\ell B_{\ell\ell'} - d_\ell B_{\ell k}[\varepsilon B_{k\ell'}]_+ - [\varepsilon B_{k\ell}]_+d_k B_{k\ell'}) q_\ell y_k^{-[\varepsilon B_{k\ell}]_+} y_{\ell'} y_k^{[\varepsilon B_{k\ell'}]_+}\\
    &=\delta_{\ell\ell'} - (d_\ell B_{\ell\ell'} + d_\ell B_{\ell k}[\varepsilon B_{k\ell'}]_+ + d_\ell [-\varepsilon B_{\ell k}]_+ B_{k\ell'}) q_\ell y_k^{-[\varepsilon B_{k\ell}]_+} y_{\ell'} y_k^{[\varepsilon B_{k\ell'}]_+}\\
    &=\delta_{\ell\ell'} - d_\ell B'_{\ell\ell'} (\tilde\tau_\cX^{k,\varepsilon})^*(q'_\ell) (\tilde\tau_\cX^{k,\varepsilon})^*(y'_{\ell'}).
  \end{align*}
  The case $\ell'=k$ in the last computation is trivial and we omit the details.

  For $\ell\ne k$, we have
  \begin{align*}
    \{(\tilde\tau_\cX^{k,\varepsilon})^*(q'_k),(\tilde\tau_\cX^{k,\varepsilon})^*(q'_\ell)\}_{\cG_\cX}
    &=\left\{-q_k y_k^2 + \sum\limits_{\ell'=1}^n[\varepsilon B_{k\ell'}]_+ q_{\ell'} y_{\ell'} y_k,q_\ell y_k^{-[\varepsilon B_{k\ell}]_+}\right\}_{\cG_\cX}\\
    &=-(d_k B_{k\ell}-2 d_k B_{k\ell}) q_k q_\ell y_k^{2-[\varepsilon B_{k\ell}]_+} + [\varepsilon B_{k\ell}]_+ q_\ell y_k^{1-[\varepsilon B_{k\ell}]_+}\\
    &\quad + \sum\limits_{\ell'\ne\ell} [\varepsilon B_{k\ell'}]_+\Big(d_{\ell'} B_{\ell' \ell} - d_{\ell'} B_{\ell' \ell} -d_k B_{k\ell}\\[-13pt]
    &\hspace{1.5in}+ d_{\ell'} B_{\ell' k} [\varepsilon B_{k\ell}]_+ - d_{\ell'} B_{\ell' k} [\varepsilon B_{k\ell}]_+ \Big) q_{\ell'} q_\ell y_{\ell'} y_k^{1-[\varepsilon B_{k\ell}]_+}\\
    &\quad + [\varepsilon B_{k\ell}]_+ \bigg(-q_\ell y_k^{1-[\varepsilon B_{k\ell}]_+} + \Big( -d_k B_{k\ell} + d_\ell B_{\ell k} [\varepsilon B_{k\ell}]_+\\[-11pt]
    &\hspace{2.53in}- d_\ell B_{\ell k} [\varepsilon B_{k\ell}]_+\Big) q_\ell^2 y_\ell y_k^{1-[\varepsilon B_{k\ell}]_+}\bigg)\\
    &=-d_k B_{k\ell} (-q_k y_k^2 + \sum\limits_{\ell'=1}^n[\varepsilon B_{k\ell'}]_+ q_{\ell'} y_{\ell'} y_k)(q_\ell y_k^{-[\varepsilon B_{k\ell}]_+})\\
    &=d_k B'_{k\ell}(\tilde\tau_\cX^{k,\varepsilon})^*(q'_k) (\tilde\tau_\cX^{k,\varepsilon})^*(q'_\ell)
  \end{align*}
  and
  \begin{align*}
    \{(\tilde\tau_\cX^{k,\varepsilon})^*(q'_k),(\tilde\tau_\cX^{k,\varepsilon})^*(y'_\ell)\}_{\cG_\cX}
    &=\left\{-q_k y_k^2 + \sum\limits_{\ell'=1}^n[\varepsilon B_{k\ell'}]_+ q_{\ell'} y_{\ell'} y_k,y_\ell y_k^{[\varepsilon B_{k\ell}]_+}\right\}_{\cG_\cX}\\
    &=-(-d_k B_{k\ell} + 2 d_k B_{k\ell}) q_k y_\ell y_k^{2+[\varepsilon B_{k\ell}]_+} - [\varepsilon B_{k\ell}]_+ y_\ell y_k^{1+[\varepsilon B_{k\ell}]_+}\\
    &\quad + \sum\limits_{\ell'\ne\ell} [\varepsilon B_{k\ell'}]_+ \Big(-d_{\ell'} B_{\ell' \ell} + d_{\ell'} B_{\ell' \ell} + d_k B_{k\ell}\\[-13pt]
    &\hspace{1.67in}- d_{\ell'} B_{\ell' k} [\varepsilon B_{k\ell}]_+ + d_{\ell'} B_{\ell' k} [\varepsilon B_{k\ell}]_+ \Big) q_{\ell'} y_{\ell'} y_\ell y_k^{1+[\varepsilon B_{k\ell}]_+}\\
    &\quad + [\varepsilon B_{k\ell}]_+ \bigg(y_\ell y_k^{1+[\varepsilon B_{k\ell}]_+} + \Big( d_k B_{k\ell} + d_\ell B_{\ell k} [\varepsilon B_{k\ell}]_+\\[-11pt]
    &\hspace{2.20in}- d_\ell B_{\ell k} [\varepsilon B_{k\ell}]_+\Big) q_\ell y_\ell^2 y_k^{1+[\varepsilon B_{k\ell}]_+}\bigg)\\
    &=d_k B_{k\ell} (-q_k y_k^2 + \sum\limits_{\ell'=1}^n[\varepsilon B_{k\ell'}]_+ q_{\ell'} y_{\ell'} y_k) (y_\ell y_k^{[\varepsilon B_{k\ell}]_+})\\
    &=-d_k B'_{k\ell}(\tilde\tau_\cX^{k,\varepsilon})^*(q'_k) (\tilde\tau_\cX^{k,\varepsilon})^*(y'_\ell).
  \end{align*}
  This completes the proof for $\tilde\tau_\cX^{k,\varepsilon}$.

  For $j,j'\ne k$, we have
  \begin{align*}
    \{(\tilde\tau_\cA^{k,\varepsilon})^*(p'_j),(\tilde\tau_\cA^{k,\varepsilon})^*(p'_{j'})\}_{\cG_\cA}
    &=\left\{p_j+[-\varepsilon B_{jk}]_+\frac{p_k x_k}{x_j},p_{j'}+[-\varepsilon B_{j'k}]_+\frac{p_k x_k}{x_{j'}}\right\}_{\cG_\cA}\\
    &=\Omega_{jj'} p_j p_{j'} + [-\varepsilon B_{jk}]_+ (\Omega_{kj'} - \Omega_{kj'} + \Omega_{jj'}) \frac{p_k x_k}{x_j} p_{j'}\\
    &\hspace{0.725in} + [-\varepsilon B_{j'k}]_+ (\Omega_{jk} - \Omega_{jk} + \Omega_{jj'}) p_j \frac{p_k x_k}{x_{j'}}\\
    &\hspace{0.725in} + [-\varepsilon B_{jk}]_+ [-\varepsilon B_{j'k}]_+ (\Omega_{kj'} - \Omega_{kj'} + \Omega_{jk} - \Omega_{jk} + \Omega_{jj'}) \frac{p_k x_k}{x_j} \frac{p_k x_k}{x_{j'}}\\
    &=\Omega_{jj'} (p_j+[-\varepsilon B_{jk}]_+\frac{p_k x_k}{x_j}) (p_{j'}+[-\varepsilon B_{j'k}]_+\frac{p_k x_k}{x_{j'}})\\
    &=\Omega'_{jj'} (\tilde\tau_\cA^{k,\varepsilon})^*(p'_j) (\tilde\tau_\cA^{k,\varepsilon})^*(p'_{j'}).
  \end{align*}
  The analogous computation with $p'_j$ replaced by $x'_j$ is straightforward and left to the reader.

  For $j\ne k$, we have
  \begin{align*}
    \{(\tilde\tau_\cA^{k,\varepsilon})^*(x'_k),(\tilde\tau_\cA^{k,\varepsilon})^*(p'_j)\}_{\cG_\cA}
    &=\left\{x_k^{-1}\prod\limits_{i=1}^m x_i^{[-\varepsilon B_{ik}]_+},p_j+[-\varepsilon B_{jk}]_+\frac{p_k x_k}{x_j}\right\}_{\cG_\cA}\\
    &=(\Omega_{kj} - \sum\limits_{i=1}^m [-\varepsilon B_{ik}]_+ \Omega_{ij}) x_k^{-1}\prod\limits_{i=1}^m x_i^{[-\varepsilon B_{ik}]_+} p_j - [-\varepsilon B_{jk}]_+ x_k^{-1} x_j^{-1} \prod\limits_{i=1}^m x_i^{[-\varepsilon B_{ik}]_+}\\
    &\quad + x_k^{-2}\prod\limits_{i=1}^m x_i^{[-\varepsilon B_{ik}]_+} [-\varepsilon B_{jk}]_+\frac{x_k}{x_j}\\
    &\quad + \bigg(\Omega_{kj} + \sum\limits_{i=1}^m [-\varepsilon B_{ik}]_+ (-\Omega_{ik} + \Omega_{ik} - \Omega_{ij}) \bigg) x_k^{-1}\prod\limits_{i=1}^m x_i^{[-\varepsilon B_{ik}]_+} [-\varepsilon B_{jk}]_+\frac{p_k x_k}{x_j}\\
    &=(\Omega_{kj} - \sum\limits_{i=1}^m [-\varepsilon B_{ik}]_+ \Omega_{ij}) (x_k^{-1}\prod\limits_{i=1}^m x_i^{[-\varepsilon B_{ik}]_+}) (p_j+[-\varepsilon B_{jk}]_+\frac{p_k x_k}{x_j})\\
    &= -\Omega'_{kj}(\tilde\tau_\cA^{k,\varepsilon})^*(x'_k) (\tilde\tau_\cA^{k,\varepsilon})^*(p'_j)
  \end{align*}
  and
  \begin{align*}
    \{(\tilde\tau_\cA^{k,\varepsilon})^*(p'_k),(\tilde\tau_\cA^{k,\varepsilon})^*(p'_j)\}_{\cG_\cA}
    &=\left\{-p_k x_k^2 \prod\limits_{i=1}^m x_i^{-[-\varepsilon B_{ik}]_+},p_j + [-\varepsilon B_{jk}]_+ \frac{p_k x_k}{x_j}\right\}_{\cG_\cA}\\
    &=-(\Omega_{kj} - 2\Omega_{kj} + \sum\limits_{i=1}^m [-\varepsilon B_{ik}]_+ \Omega_{ij}) p_k x_k^2 \prod\limits_{i=1}^m x_i^{-[-\varepsilon B_{ik}]_+} p_j\\
    &\quad - [-\varepsilon B_{jk}]_+ p_k x_k^2 x_j^{-1} \prod\limits_{i=1}^m x_i^{-[-\varepsilon B_{ik}]_+} - x_k^2 \left(\prod\limits_{i=1}^m x_i^{-[-\varepsilon B_{ik}]_+}\right) [-\varepsilon B_{jk}]_+ \frac{p_k}{x_j}\\
    &\quad - \bigg(\Omega_{kj} - 2\Omega_{kj} - \sum\limits_{i=1}^m [-\varepsilon B_{ik}]_+ (-\Omega_{ik} + \Omega_{ik} - \Omega_{ij}) \bigg) p_k x_k^2 \left(\prod\limits_{i=1}^m x_i^{-[-\varepsilon B_{ik}]_+}\right)\times\\
    &\hspace{3.9in} \times[-\varepsilon B_{jk}]_+ \frac{p_k x_k}{x_j}\\
    &\quad + 2 p_k x_k \left(\prod\limits_{i=1}^m x_i^{-[-\varepsilon B_{ik}]_+}\right) [-\varepsilon B_{jk}]_+ \frac{x_k}{x_j}\\
    &=(-\Omega_{kj} + \sum\limits_{i=1}^m [-\varepsilon B_{ik}]_+ \Omega_{ij}) (-p_k x_k^2 \prod\limits_{i=1}^m x_i^{-[-\varepsilon B_{ik}]_+}) (p_j + [-\varepsilon B_{jk}]_+ \frac{p_k x_k}{x_j})\\
    &=\Omega'_{kj}(\tilde\tau_\cA^{k,\varepsilon})^*(p'_k) (\tilde\tau_\cA^{k,\varepsilon})^*(p'_j)
  \end{align*}
  and
  \begin{align*}
    \{(\tilde\tau_\cA^{k,\varepsilon})^*(p'_k),(\tilde\tau_\cA^{k,\varepsilon})^*(x'_j)\}_{\cG_\cA}
    &=\left\{-p_k x_k^2 \prod\limits_{i=1}^m x_i^{-[-\varepsilon B_{ik}]_+},x_j\right\}_{\cG_\cA}\\
    &= -(-\Omega_{kj} + 2\Omega_{kj} - \sum\limits_{i=1}^m [-\varepsilon B_{ik}]_+ \Omega_{ij}) p_k x_k^2\left(\prod\limits_{i=1}^m x_i^{-[-\varepsilon B_{ik}]_+}\right)x_j\\
    &= -\Omega'_{kj}(\tilde\tau_\cA^{k,\varepsilon})^*(p'_k) (\tilde\tau_\cA^{k,\varepsilon})^*(x'_j).
  \end{align*}
  Finally, we have
  \begin{align*}
    \{(\tilde\tau_\cA^{k,\varepsilon})^*(p'_k),(\tilde\tau_\cA^{k,\varepsilon})^*(x'_k)\}_{\cG_\cA}
    &=\left\{-p_k x_k^2\left(\prod\limits_{i=1}^m x_i^{-[-\varepsilon B_{ik}]_+}\right),x_k^{-1}\prod\limits_{i=1}^m x_i^{[-\varepsilon B_{ik}]_+}\right\}_{\cG_\cA}\\
    &=x_k^2\left(\prod\limits_{i=1}^m x_i^{-[-\varepsilon B_{ik}]_+}\right)x_k^{-2}\prod\limits_{i=1}^m x_i^{[-\varepsilon B_{ik}]_+}\\
    &\quad -\bigg(\sum\limits_{i=1}^m \Big([-\varepsilon B_{ik}]_+ (-\Omega_{ki} + 2\Omega_{ki}) + [-\varepsilon B_{ik}]_+ \Omega_{ik}\Big) \bigg) p_k x_k^2 \left(\prod\limits_{i=1}^m x_i^{-[-\varepsilon B_{ik}]_+}\right)\times\\
    &\hspace{3.7in}\times x_k^{-1}\prod\limits_{i=1}^m x_i^{[-\varepsilon B_{ik}]_+}\\
    &=1.
  \end{align*}
  This completes the proof for $\tilde\tau_\cA^{k,\varepsilon}$.
\end{proof}
\newpage

\begin{itemize}
\item Find the time-1 flow of the multiplicative Hamiltonian vector field on the groupoids.  Done.
\item Lift the canonical change of variables to groupoids.
\item Composition of the two groupoid maps give us the groupoid mutations.
\end{itemize}

\newpage 

To record the iteration of mutations, we introduce the $n$-regular rooted tree $\TT_n$ with root vertex $t_0$ and with the $n$ edges emanating from each vertex labeled by the set $\{1,\ldots,n\}$.
In particular, each vertex $t$ of $\TT_n$ is uniquely determined by a sequence of indices specifying the edge labels along the unique path from $t_0$ to $t$.
Fix an initial $m\times n$ exchange matrix $\tilde B_{t_0}$ and assign exchange matrices $\tilde B_t$ to the vertices $t\in\TT_n$ so that $\tilde B_{t'}=\mu_k\tilde B_t$ whenever $t$ and $t'$ are joined by an edge labeled by $k$.
The collection $\{\tilde B_t\}_{t\in\TT_n}$ is called the \emph{mutation pattern} generated by $\tilde B_{t_0}$ and any two exchange matrices $\tilde B_t$, $\tilde B_{t'}$ for $t,t'\in\TT_n$ are said to be \emph{mutation equivalent}.
Given an $m\times m$ matrix $\Omega_{t_0}$ compatible with $\tilde B_{t_0}$, we define matrices $\Omega_t$ compatible with $\tilde B_t$ by iterating mutations as above.

The mutation pattern gives rise to the following important combinatorial construction.
\begin{example}
  Given an $n\times n$ skew-symmetrizable matrix $B$, let $\tilde B_{t_0}=\tilde B_{prin}$ denote the $2n\times n$ exchange matrix with principal submatrix $B$ and lower $n\times n$ submatrix given by the $n\times n$ identity matrix $I_n$.
  %In this case a seed $\Sigma_{prin}=(\bfx_{prin},\tilde B_{prin})$ is said to have \emph{principal coefficients}.
  Given $\tilde B_t$ mutation equivalent to $\tilde B_{t_0}$, the lower $n\times n$ submatrix $C_t$ of $\tilde B_t$ has the following remarkable \emph{sign-coherence} property: each column of $C_t$, known as a $\bfc$-vector, has either all non-negative entries or all non-positive entries \cite{fomin-zelevinsky4,nakanishi-zelevinsky,gross-hacking-keel-kontsevich}.
  Thus each exchange matrix $\tilde B_t$ mutation equivalent to $\tilde B_{t_0}$ admits a collection of \emph{tropical signs} $\varepsilon_{k;t}\in\{\pm1\}$, where $\varepsilon_{k;t}=1$ if the entries in the $k$-th column of $C_t$ are non-negative and $\varepsilon_{k;t}=-1$ if the entries in the $k$-th column of $C_t$ are non-positive. 
\end{example}

\begin{remark}
  Given an arbitrary exchange matrix $\tilde B$, we may construct a corresponding principalized exchange matrix $\tilde B_{prin}$ from the principal submatrix $B$.
  Then for each vertex $t\in\TT_n$, we associate the same tropical signs $\varepsilon_{k;t}$ to the columns of $\tilde B_t$.
\end{remark}

The mutation of exchange matrices is accompanied by two notions of mutation on rational functions.
Let $\cF$ and $\cP$ be extension fields of $\QQ$ with transcendence degrees $m$ and $n$ respectively.   
A \emph{compatible seed} is a quadruple $\Sigma=(\bfx,\bfy,\tilde B,\Omega)$, where
\begin{itemize}
  \item $\bfx=(x_1,\ldots,x_m)$ is a transcendence basis of $\cF$ over $\QQ$ with entries called \emph{cluster variables};
  \item $\bfy=(y_1,\ldots,y_n)$ is a transcendence basis of $\cP$ over $\QQ$ with entries called \emph{coefficient variables};
  \item $\tilde B$ is an $m\times n$ exchange matrix;
  \item $\Omega$ is an $m\times m$ matrix compatible with $\tilde B$.
\end{itemize}
For $1\le k\le n$ and a sign $\varepsilon\in\{\pm1\}$, define the \emph{mutation of $\Sigma$ in direction $k$} by $\mu_{k,\varepsilon}\Sigma=(\mu_{k,\varepsilon}\bfx,\mu_{k,\varepsilon}\bfy,\mu_k\tilde B,\mu_k\Omega)$, where 
\begin{itemize}
  \item $\mu_{k,\varepsilon}\bfx=(x'_1,\ldots,x'_m)$ is given by the exchange relation
    \begin{equation}
      \label{eq:x exchange relation}
      x'_i=\begin{cases} x_i & \text{if $i\ne k$;}\\ x_k^{-1}\prod\limits_{i=1}^m x_i^{[-\varepsilon B_{ik}]_+}\left(1+\prod\limits_{i=1}^m x_i^{\varepsilon B_{ik}}\right) & \text{if $i=k$;}\end{cases}
    \end{equation}
  \item $\mu_{k,\varepsilon}\bfy=(y'_1,\ldots,y'_n)$ is given by the exchange relation 
    \begin{equation}
      \label{eq:y exchange relation}
      y'_j=\begin{cases} y_j^{-1} & \text{if $j=k$;}\\ y_jy_k^{[\varepsilon B_{kj}]_+}(1+y_k^\varepsilon)^{-B_{kj}} & \text{if $j\ne k$.}\end{cases}
    \end{equation}
\end{itemize}
Observe that seed mutation is again involutive and that there is a map $\pi:\cP\to\cF$ given by 
\begin{equation}
  \label{eq:A to X}
  \pi(y_k)=\hat y_k:=\prod\limits_{i=1}^m x_i^{B_{ik}}
\end{equation}
which is compatible with mutations.

Given an initial seed $\Sigma_{t_0}=(\bfx_{t_0},\bfy_{t_0},\tilde B_{t_0},\Omega_{t_0})$, we label seeds $\Sigma_t=(\bfx_t,\bfy_t,\tilde B_t,\Omega_t)$ which are mutation equivalent to $\Sigma_{t_0}$ by vertices of $\TT_n$ as above, here we write $\bfx_t=(x_{1;t},\ldots,x_{m;t})$ and $\bfy_t=(y_{1;t},\ldots,y_{m;t})$.
The resulting collection $\{\Sigma_t\}_{t\in\TT_n}$ is known as the \emph{exchange pattern} generated by $\Sigma_{t_0}$.
\begin{definition}
  Let $\Sigma$ be a seed in $\cF$.
  The \emph{cluster algebra} $\cA(\Sigma)$ is the $\ZZ$-subalgebra of $\cF$ generated by all cluster variables $x_{i;t}$ from seeds $\Sigma_t$ mutation equivalent to $\Sigma_{t_0}$.
\end{definition}
By iterating the exchange relations we appear to get elements of $\QQ(x_1,\ldots,x_m)\subset\cF$, that is rational functions in $x_1,\ldots,x_m$.  
The following result of Fomin and Zelevinsky known as ``the Laurent phenomenon'' shows that the cluster variables always take on a much simpler form.
\begin{theorem}
  \cite{fomin-zelevinsky1}
  Let $\Sigma$ be a seed in $\cF$ and $\Sigma_t$ any seed in the exchange pattern generated by $\Sigma$.
  Each cluster variable $x_{i;t}$ of $\Sigma_t$ is an element of the subring $\ZZ[x_1^{\pm1},\ldots,x_m^{\pm1}]\subset\cF$.
\end{theorem}

In fact, the situtation is even better:\footnote{do we need this?} the initial cluster Laurent expansions of all cluster variables have positive integer coefficients.
\begin{theorem}\cite{lee-schiffler, gross-hacking-keel-kontsevich}
  Let $\Sigma$ be a seed in $\cF$ and $\Sigma'\sim\Sigma$.  Each cluster variable $x'_i$ of $\Sigma'$ is an element of the subsemiring $\ZZ_{\ge0}[x_1^{\pm1},\ldots,x_m^{\pm1}]\subset\cF$. 
\end{theorem}
For $x'_i$ a cluster variable from a seed $\Sigma'\sim\Sigma$, we write $x'_i(\bfx)$ when we wish to emphasize that $x'_i$ should be thought of as a function of the cluster variables in $\bfx=(x_1,\ldots,x_m)$.

\subsection{The Cluster Manifold and Compatible Poisson Structures}
Fix the field $\FF=\RR$ or $\FF=\CC$.
For an $m\times n$ exchange matrix $\tilde B$, define the \emph{cluster chart} $\Spec(\FF[x_1^{\pm1},\ldots,x_m^{\pm1}])$.
Observe that $\Sigma=(\bfx,\tilde B)$ is a seed in the field of rational functions on this cluster chart and thus we denote it by $L_\Sigma$.
Then the exchange relation \eqref{eq:exchange relations} provides a birational transformation between the cluster charts $\varphi_{\Sigma,\mu_k\Sigma}:L_\Sigma\to L_{\mu_k\Sigma}$ for $1\le k\le n$.
By composing these \emph{elementary transition maps} for neighboring seeds we get a birational transformation between $\varphi_{\Sigma,\Sigma'}:L_\Sigma\to L_{\Sigma'}$ for any seeds $\Sigma\sim\Sigma'$.

Given any seed $\Sigma$, the transition maps above define the \emph{cluster manifold} $M=M(\Sigma)=\bigcup\limits_{\Sigma'\sim\Sigma}L_{\Sigma'}$.
By construction we have $\cA(\Sigma)\subset C^\infty(M)$ and any Poisson structure on $\cA(\Sigma)$ naturally extends to give a Poisson structure on $C^\infty(M)$.

\begin{definition}
  A Poisson structure $\{\cdot,\cdot\}:\cA(\Sigma)\times\cA(\Sigma)\to\cA(\Sigma)$ is \emph{compatible} with the cluster algebra structure if, for each seed $\Sigma'\sim\Sigma$, the cluster variables in $\bfx'$ are \emph{log-canonical} with respect to $\{\cdot,\cdot\}$.
  That is, there exists a skew-symmetric integer \emph{coefficient matrix} $\Omega'=(\Omega'_{ij})$ so that 
  \begin{equation}
    \label{eq:log-canonical bracket}
    \{x'_i,x'_j\}=\Omega'_{ij}x'_ix'_j
  \end{equation}
  for $1\le i,j\le m$.
\end{definition}
\begin{remark}
  Suppose the cluster variables of a seed $\Sigma=(\bfx,\tilde B)$ are log-canonical with respect to a Poisson bracket $\{\cdot,\cdot\}:\cA(\Sigma)\times\cA(\Sigma)\to\cA(\Sigma)$ with coefficient matrix $\Omega$.
  Then the compatibility of $\{\cdot,\cdot\}$, together with the exchange relations, imposes the compatibility condition $\tilde B^T\Omega=[D\ 0]$ (see \cite{berenstein-zelevinsky,gekhtman-shapiro-vainshtein} for details).
\end{remark}

\begin{theorem}
  \cite{gekhtman-shapiro-vainshtein}
  Suppose the $m\times n$ exchange matrix $\tilde B$ of a seed $\Sigma$ has full rank.  Then there exists a Poisson structure $\Omega$ compatible with the cluster structure on $\cA(\Sigma)$.
\end{theorem}





%%%%%%%%%%%%%%%%%%%%%%%%%%
\section{Cluster Symplectic Groupoids}
Let $\Sigma=(\bfx,\tilde B,\Theta)$ be a graded seed and assume there exists a compatible Poisson structure on $L_\Sigma$ with coefficient matrix $\Omega=(\Omega_{ij})$.  
%Write $\pi$ for the corresponding Poisson bivector on the cluster manifold $M$, i.e.\ in local coordinates on $L_\Sigma$ we have $\pi=\sum\limits_{i>j}\Omega_{ij}x_ix_j\frac{\partial}{\partial x_i}\wedge\frac{\partial}{\partial x_j}$.
In this section, we give an integration to a symplectic groupoid $G(\Sigma)$ of the Poisson structure on a cluster manifold $M(\Sigma)$.  

We build the cluster symplectic groupoid $G(\Sigma)\rightrightarrows M(\Sigma)$ by gluing together local groupoid charts $G_{\Sigma'}\rightrightarrows L_{\Sigma'}$, $\Sigma'\sim\Sigma$, along transition maps which lift the cluster mutations used to glue cluster charts of $M(\Sigma)$.
This process is carried out in three steps:
\begin{itemize}
  \item first, we show that the action groupoids $(\FF^*)^m\times L_{\Sigma'}\rightrightarrows L_{\Sigma'}$ over each cluster chart admit a gluing which lifts the cluster mutation;
  \item second, we define maps $T^*L_{\Sigma'}\to(\FF^*)^m\times L_{\Sigma'}$ along which we pullback the groupoid structure to obtain symplectic groupoids integrating a compatible Poisson structure on $L_{\Sigma'}$;
  \item finally, we define transition maps between the symplectic groupoids $G_{\Sigma'}=T^*L_{\Sigma'}$ which lift the cluster mutations.
\end{itemize}

There is a natural action groupoid structure $(\FF^*)^m\times L_\Sigma\rightrightarrows L_\Sigma$ with source map $\alpha$ being the natural projection and target map given by the Hadamard product
\[\beta(\bfs,\bfx)=\bfs\circ\bfx,\quad \bf\circ\bfx=(s_1x_1,\ldots,s_mx_m),\]
i.e.\ given by the natural action of $(\FF^*)^m$ on $L_\Sigma$.

\erase{
Given any seed $\Sigma'\sim\Sigma$, define a map $\mu_{\Sigma',\Sigma}:(\FF^*)^r\times L_\Sigma\to(\FF^*)^r\times L_\Sigma'$ by $\mu_{\Sigma',\Sigma}(\bfs,\bfx)=(\bfs',\bfx')$, where $\bfx'(\bfx)=\big(x'_1(\bfx),\ldots,x'_m(\bfx)\big)$ and $\bfs'(\bfs,\bfx)=(s'_1(\bfs,\bfx),\ldots,s'_m(\bfs,\bfx))$ is given by $s'_i(\bfs,\bfx)=\frac{x'_i(\bfs\circ\bfx)}{x'_i(\bfx)}$.
\begin{theorem}
  For any three seeds $\Sigma\sim\Sigma'\sim\Sigma''$, we have $\mu_{\Sigma'',\Sigma'}\mu_{\Sigma',\Sigma}=\mu_{\Sigma'',\Sigma}$ and hence the local groupoid charts glue to give a groupoid over the cluster manifold $M(\Sigma)$.
\end{theorem}
\begin{proof}
  By induction, it suffices to prove the claim when $\Sigma''=\mu_k\Sigma'$ for some $k$.
  In this case, we have $x''_i(\bfx)=x'_i(\bfx)$ and thus $s''_i(\bfs,\bfx)=s'_i(\bfs,\bfx)$ for $i\ne k$.
  Observe that the definition of $\mu_{\Sigma',\Sigma}$ gives $\bfs'(\bfs,\bfx)\circ\bfx'(\bfx)=\bfx'(\bfs\circ\bfx)$ and the definition of $\mu_k$ gives $\bfx''(\bfx'(\bfx))=\bfx''(\bfx)$.
  It then immediately follows from the definition of $\mu_{\Sigma'',\Sigma'}\mu_{\Sigma',\Sigma}$ that we have
  \[s''_k(\bfs'(\bfs,\bfx),\bfx'(\bfx))=\frac{x''_k(\bfs'(\bfs,\bfx)\circ\bfx'(\bfx))}{x''_k(\bfx'(\bfx))}=\frac{x''_k(\bfx'(\bfs\circ\bfx))}{x''_k(\bfx'(\bfx))}=\frac{x''_k(\bfs\circ\bfx)}{x''_k(\bfx)}=s''_k(\bfs,\bfx).\]
\end{proof}}%end erase

Let $G_\Sigma=T^*L_\Sigma$ denote the cotangent bundle of $L_\Sigma$.
Write $\bfp=(p_1,\ldots,p_m)$ for the cotangent coordinates of $G_\Sigma$.
Define a map $\rho_\Sigma:G_\Sigma\to(\FF^*)^m\times L_\Sigma$ by $\rho_\Sigma(\bfx,\bfp)=(\bfs(\bfx,\bfp),\bfx)$, with $s_i(\bfx,\bfp)=e^{\sum_j\Omega_{ij}x_jp_j}$.

%Define the map $\beta:G_\Sigma\to L_\Sigma$ by 
%\begin{equation}
%  \label{eq:cluster groupoid target map}
%  \beta(\bfx,\bfp)=(s_1x_1,\ldots,s_nx_n),\quad\text{where}\quad s_i:=e^{\sum_j\Omega_{ij}x_jp_j}.
%\end{equation}
\begin{theorem}
  \label{th:cluster groupoid}
  The groupoid structure on $(\FF^*)^m\times L_\Sigma$ pulls back to a groupoid structure on the manifold $G_\Sigma$ with source map the natural projection, target map $\beta\circ\rho_\Sigma$, multiplication given by
  \[(\bfx,\bfp)\cdot\big((\beta\circ\rho_\Sigma)(\bfx,\bfp),\bfp'\big)=(\bfx,\bfp''),\quad p''_i=s_i(\bfx,\bfp)p'_i+p_i,\]
  inversion given by
  \[(\bfx,\bfp)^{-1}=\big(\beta(\bfx,\bfp),\bfp'),\quad p'_i=-s_i(\bfx,\bfp)^{-1}p_i,\]
  and identity map given by $1_\bfx=(\bfx,\boldsymbol{0})$.
  %Moreover, the 2-form $\omega=\alpha^*(\pi^{-1})-\beta^*(\pi^{-1})$ defines a symplectic structure on $G_\Sigma$.
\end{theorem}
%\begin{proof}
%  It is clear that $1_\bfx$ gives the identity map for this multiplication and that the inversion map satisfies 
%  \[(\bfx,\bfp)\cdot(\bfx,\bfp)^{-1}=1_\bfx=(\bfx,\bfp)^{-1}\cdot(\bfx,\bfp)\]
%  for all $(\bfx,\bfp)\in G_\Sigma$.
%
%  It remains to check associativity of the multiplication.
%  Fix an element $(\bfx,\bfp)\in G_\Sigma$.
%  Consider $(\bfx',\bfp'),(\bfx'',\bfp'')\in G_\Sigma$ with $\bfx'=\beta(\bfx,\bfp)$ and $\bfx''=\beta(\bfx',\bfp')$.
%  Then we have
%  \[\big((\bfx,\bfp)\cdot(\bfx',\bfp')\big)\cdot(\bfx'',\bfp'')=(\bfx,\bfp'''),\quad p'''_i=e^{\sum_j\Omega_{ij}x_j(s_jp'_j+p_j)}p''_i+e^{\sum_j\Omega_{ij}x_jp_j}p'_i+p_i.\]
%  On the other hand we have
%  \[(\bfx,\bfp)\cdot\big((\bfx',\bfp')\cdot(\bfx'',\bfp'')\big)=(\bfx,\bfp'''),\quad p'''_i=e^{\sum_j\Omega_{ij}x_jp_j}(e^{\sum_j\Omega_{ij}x'_jp'_j}p''_i+p'_i)+p_i\]
%  and thus associativity holds.
%\end{proof}

Write $\mu_k\Sigma=(\bfx',\tilde B')$.  
%Let $\bfp'=(p'_1,\ldots,p'_n)$ denote the cotangent coordinates of $G_{\mu_k\Sigma}$.
Define a map from $G_\Sigma$ to $G_{\mu_k\Sigma}$, which we also denote $\mu_{\Sigma,\mu_k\Sigma}$, as follows:
\begin{equation}
  \label{eq:groupoid gluing map}
  \mu_{\Sigma,\mu_k\Sigma}(\bfx,\bfp)=(\bfx'(\bfx),\bfp'(\bfx,\bfp)),\quad \bfp'(\bfx,\bfp)=(p'_1(\bfx,\bfp),\ldots,p'_m(\bfx,\bfp)),\quad p'_\ell(\bfx,\bfp)=\frac{x_\ell p_\ell +[\varepsilon_k b_{\ell k}]_+ x_k p_k +\frac{b_{\ell k}}{d_k}\ln\left(\frac{Q_k(\bfs\circ\bfx)}{Q_k(\bfx)}\right)}{x'_\ell(\bfx)},
\end{equation}
where $\varepsilon_k$ denotes the tropical sign for the seed $\Sigma$.

\begin{lemma}
  Let $\bfx',\bfp',\Omega'$ be obtained from $\bfx,\bfp,\Omega$ by mutation in direction $k$.
  For any index $1\le i\le m$ with $i\ne k$, we have $\sum\limits_{j=1}^m \Omega'_{ij} x'_j p'_j = \sum\limits_{j=1}^m \Omega_{ij} x_j p_j$ and $\sum\limits_{j=1}^m \Omega'_{kj} x'_j p'_j=\ln\left(\frac{x'_k(\bfs\circ\bfx)}{x'_k(\bfx)}\right)$.
\end{lemma}
\begin{proof}
  The mutation operation for groupoid charts is more naturally written in vector form as
  \[\bfx'\circ\bfp'=E_{k,\varepsilon_k}(\bfx\circ\bfp)+\frac{1}{d_k}\ln\left(\frac{Q_k(\bfs\circ\bfx)}{Q_k(\bfx)}\right)\bfb_k.\]
  This gives rise to the identity
  \begin{align}
    \Omega'(\bfx'\circ\bfp')
    \nonumber &=\Omega' E_{k,\varepsilon_k}(\bfx\circ\bfp)+\frac{1}{d_k}\ln\left(\frac{Q_k(\bfs\circ\bfx)}{Q_k(\bfx)}\right)\Omega'\bfb_k\\
    \label{eq:groupoid transition} &=E_{k,\varepsilon_k}^T\Omega (\bfx\circ\bfp)+\ln\left(\frac{Q_k(\bfs\circ\bfx)}{Q_k(\bfx)}\right)\bfe_k.
  \end{align}
  By the structure of $E_{k,\varepsilon_k}$, the euqality $\sum\limits_{j=1}^m \Omega'_{ij} x'_j p'_j = \sum\limits_{j=1}^m \Omega_{ij} x_j p_j$ for $i\ne k$ immediately follows.

  Equation~\ref{eq:groupoid transition} also gives the identity
  \begin{align*}
    \sum\limits_{j=1}^m \Omega'_{kj} x'_j p'_j
    &=-\sum\limits_{j=1}^m \Omega_{kj} x_j p_j+\ln\left(\bfs^{\bfb_k^+}\right)+\ln\left(\frac{Q_k(\bfs\circ\bfx)}{Q_k(\bfx)}\right)\\
    &=-\sum\limits_{j=1}^m \Omega_{kj} x_j p_j+\ln\left(\frac{x'_k(\bfs\circ\bfx)x_k(\bfs\circ\bfx)}{x'_k(\bfx)x_k(\bfx)}\right)\\
    \ln\left(\frac{x'_k(\bfs\circ\bfx)}{x'_k(\bfx)}\right).
  \end{align*}
\end{proof}

\begin{lemma}
  The mutation of cluster groupoid charts is involutive.
\end{lemma}
\begin{proof}
  
\end{proof}


%%%%%%%%%%%%%%%%%%%%%%%%%%%%%%%%%%%%%%%%%%%%
\section{Totally Positive Cluster Manifolds}
In this section we show that the totally nonnegative part $M_{\ge0}(\Sigma)$ of a cluster manifold is a manifold with corners in the sense of \cite{MR3077259}.  
Moreover, we show that the nonnegative cluster manifold is a union of symplectic leaves for any compatible Poisson structure on $\cA(\Sigma)$.  
The symplectic leaves of $M_{\ge0}(\Sigma)$ are naturally labelled by compatible subsets of cluster variables, where the number of cluster variables in the labeling set determines the corank of the symplectic leaf.
Here there is a unique dense symplectic leaf and the boundary of $M_{\ge0}(\Sigma)$ is again a union of symplectic leaves of lower dimension where the Poisson structure degenerates.

\begin{theorem}
  Let $\Sigma$ be a seed.  
  The 1-skeleton of $M_{\ge0}(\Sigma)$ given by 0-dimensional and 1-dimensional symplectic leaves identifies with the exchange graph of $\cA(\Sigma)$.  
  Moreover, if $\Sigma$ is a seed of finite-type, then $M_{\ge0}(\Sigma)$ provides a realization of the generalized associahedron with the same Cartan type as $\Sigma$.
\end{theorem}
\begin{proof}
  The 0-dimensional symplectic leaves correspond to the vanishing of all cluster variables from a seed mutation equivalent to $\Sigma$.  
  Then a 1-dimensional symplectic leaf whose boundaries correspond to seeds $\Sigma'$ and $\Sigma''$ exactly corresponds to the non-vanishing of exchangable cluster variables $x'_k$ and $x''_k$.
  But this is exactly the exchange graph of $\cA(\Sigma)$.

  When $\Sigma$ is of finite-type, the realization of $M_{\ge0}(\Sigma)$ as a simplicial complex, given by taking symplectic leaves as cells, is naturally dual to the cluster complex of $\cA(\Sigma)$, i.e. $M_{\ge0}(\Sigma)$ identifies with the associated generalized associahedron.
\end{proof}

%%%%%%%%%%%%%%%%%%%%%%%%%%%%%%%%%%%%%%%%%%%%%%%%%%%%%%%%%%%%%%%%
\section{Symplectic Topology of the Nonnegative Cluster Groupoid}
Let $\cG_{\ge0}(\Sigma)$ denote the symplectic groupoid over $M_{\ge0}(\Sigma)$.  
In this section we introduce a Poisson spray which may be used to construct $\cG_{\ge0}(\Sigma)$ and apply a Moser argument to show that up to symplectomorphism $\cG_{\ge0}(\Sigma)$ can be identified with the natural symplectic structure on the cotangent bundle $T^*M_{\ge0}(\Sigma)$.

Let $\cA(\Sigma)$ be a cluster algebra of rank $n$ generated by the seed $\Sigma=(\bfx,\tilde B)$, and we assume there exists a compatible Poisson structure on $L_\Sigma$ with coefficient matrix $\Omega=(\Omega_{ij})$. That is,  $\tilde B^T\Omega=[D\ 0]$, where $D$ is a skew-symmetrizing matrix for the upper $n\times n$ submatrix $B$ of $\tilde B$, and $L_\Sigma$ is equipped with the Poisson structure
\begin{equation}
	\pi=\sum\limits_{1 \leq j \leq n, j < i \leq n}\Omega_{ij}x_ix_j\frac{\partial}{\partial x_i}\wedge\frac{\partial}{\partial x_j}.
\end{equation}
By Remark~\ref{remark:PoiSpLoc}, we may choose the Poisson spray $X$ on $L_\Sigma$:
\begin{equation}
	X = \sum\limits_{1 \leq j \leq n, j < i \leq n}\Omega_{ij}x_ip_i x_j\frac{\partial}{\partial x_j} \quad - \sum\limits_{1 \leq j \leq n, j < i \leq n}\Omega_{ij}p_ix_i p_j\frac{\partial}{\partial p_j}.
\end{equation}

\bibliographystyle{hyperamsplain}
\bibliography{cluster_symplectic}

\end{document}
