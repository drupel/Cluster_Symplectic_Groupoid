\documentclass{amsart}
\usepackage{amsmath,amsfonts,amssymb,latexsym}
\usepackage[margin=1in]{geometry}

\newtheorem{theorem}{Theorem}[section]
\newtheorem{corollary}[theorem]{Corollary}
\newtheorem{definition}[theorem]{Definition}
\newtheorem{lemma}[theorem]{Lemma}
\newtheorem{question}[theorem]{Question}
\newtheorem{proposition}[theorem]{Proposition}


\newcommand{\bfx}{\mathbf{x}}

\newcommand{\cA}{\mathcal{A}}
\newcommand{\cF}{\mathcal{F}}
\newcommand{\cG}{\mathcal{G}}

\newcommand{\CC}{\mathbb{C}}
\newcommand{\kk}{\Bbbk}
\newcommand{\QQ}{\mathbb{Q}}
\newcommand{\RR}{\mathbb{R}}
\newcommand{\ZZ}{\mathbb{Z}}

\newcommand{\diag}{\operatorname{diag}}
\renewcommand{\max}{\operatorname{max}}

\title{Symplectic Groupoids for Cluster Manifolds}

\begin{document}
\begin{abstract}
  We prove some things.
  And we are happy!
  It seems to work!
\end{abstract}
\maketitle
Outline
\begin{enumerate}
	\item Intro to Poisson manifolds, symplectic groupoid and Poisson spray
	\item Intro to cluster algebra and compatible Poisson structures 
	\item Cluster symplectic groupoid
	\item Totally positive cluster manifolds (definition of manifolds with corners [check D Joyce], associahedron of type A and generalized associahedron)
	\item Symplectic topology of the groupoid, and examples
\end{enumerate}

%%%%%%%%%%%%%%%%%%%%%%
\section{Introduction}


%%%%%%%%%%%%%%%%%%%%%%%%%%
\section{Poisson Geometry}
In this section we recall the various incarnations of Poisson manifolds and the construction of symplectic groupoids from the Poisson spray \cite{MR2900786}.

\begin{definition}
A smooth Poisson manifold is a smooth manifold $M$ with Lie bracket
$$
	\{\cdot, \cdot\}: C^\infty(M) \times C^\infty(M) \to C^\infty(M)
$$
such that the Leibniz rule
$$
	\{fg, h\} = f\{g,h\} + g\{f,h\}
$$
is satisfied. A holomorphic Poisson manifold is analogous in that $M$ is a holomorphic manifold and we replace the space of smooth functions $C^\infty(M)$ with the space of holomorphic functions $\mathcal{O}(M)$.
\end{definition}


%%%%%%%%%%%%%%%%%%%%%%%%%%
\section{Cluster Algebras}

Let $\tilde B=(b_{ij})$ be an $m\times n$ integer matrix with $m\ge n$.  
Write $B$ for the upper $n\times n$ submatrix of $\tilde B$ and assume $B$ is skew-symmetrizable, i.e. there exists an integer diagonal matrix $D=\diag(d_1,\ldots,d_n)$ with each $d_i>0$ so that $DB$ is skew-symmetric. 
We call such a matrix $\tilde B$ an $m\times n$ \emph{exchange matrix}.
For $1\le k\le n$, define the \emph{mutation of $\tilde B$ in direction $k$} by $\mu_k\tilde B=(b'_{ij})$ where
\[b'_{ij}=\begin{cases}-b_{ij} & \text{if $i=k$ or $j=k$;}\\ b_{ij}+[b_{ik}]_+b_{kj}+b_{ik}[-b_{kj}]_+ & \text{otherwise.}\end{cases}\]
Above we used the notation $[b]_+=\max\{b,0\}$.

Let $\cF$ be an extension field of $\QQ$ of transcendence degree $m$.   
A \emph{seed} in $\cF$ is a pair $\Sigma=(\bfx,\tilde B)$ where $\bfx=(x_1,\ldots,x_m)$ is a transendence basis of $\cF$ over $\QQ$ called the \emph{cluster} with entries called \emph{cluster variables} and $\tilde B$ is an $m\times n$ exchange matrix.
For $1\le k\le n$, define the \emph{mutation of $\Sigma$ in direction $k$} by $\mu_k\Sigma=(\bfx',\mu_k\tilde B)$ where 
\[\bfx'=(x'_1,\ldots,x'_m),\qquad x'_i=\begin{cases} x_i & \text{if $i\ne k$;}\\ \frac{1}{x_k}\left(\prod\limits_{i=1}^m x_i^{[b_{ik}]_+}+\prod\limits_{i=1}^m x_i^{[-b_{ik}]_+}\right) & \text{if $i=k$.}\end{cases}\]
Observe that seed mutation is involutive, i.e. $\mu_k\mu_k\Sigma=\Sigma$.
A seed $\Sigma'$ is \emph{mutation equivalent} to $\Sigma$ if there exists a sequence of mutations which transforms $\Sigma$ into $\Sigma'$, in this case we write $\Sigma'\sim\Sigma$.
\begin{definition}
  Let $\Sigma$ be a seed in $\cF$.  The \emph{cluster algebra} is the $\ZZ$-subalgebra of $\cF$ generated by all cluster variables from seeds $\Sigma'$ mutation equivalent to $\Sigma$.
\end{definition}
By iterating the exchange relations we appear to get elements of $\QQ(x_1,\ldots,x_m)\subset\cF$, that is rational functions in $x_1,\ldots,x_m$.  
The following result of Fomin and Zelevinsky known as ``the Laurent phenomenon'' shows that the cluster variables always take on a much simpler form.
\begin{theorem}\cite{fomin-zelevinsky1}
  Let $\Sigma$ be a seed in $\cF$ and $\Sigma'\sim\Sigma$.  Each cluster variable $x'_i$ of $\Sigma'$ is an element of the subring $\ZZ[x_1^{\pm1},\ldots,x_m^{\pm1}]\subset\cF$.
\end{theorem}

In fact, the situtation is even nicer: the initial cluster Laurent expansions of all cluster variables have positive integer coefficients.
\begin{theorem}\cite{lee-schiffler, gross-hacking-keel-kontsevich}
  Let $\Sigma$ be a seed in $\cF$ and $\Sigma'\sim\Sigma$.  Each cluster variable $x'_i$ of $\Sigma'$ is an element of the subsemiring $\ZZ_{\ge0}[x_1^{\pm1},\ldots,x_m^{\pm1}]\subset\cF$. 
\end{theorem}


%%%%%%%%%%%%%%%%%%%%%%%%%%%%%%%%%%%%%%
\section{Cluster Symplectic Groupoids}
In this section we give an integration to a symplectic groupoid $\cG(\Sigma)$ of the Poisson structure on a cluster manifold $M(\Sigma)$.  

%%%%%%%%%%%%%%%%%%%%%%%%%%%%%%%%%%%%%%%%%%%%
\section{Totally Positive Cluster Manifolds}
In this section we show that the totally nonnegative part $M_{\ge0}(\Sigma)$ of a cluster manifold is a manifold with corners in the sense of \cite{MR3077259}.  
Moreover, we show that the nonnegative cluster manifold is a union of symplectic leaves for any compatible Poisson structure on $\cA(\Sigma)$.  
The symplectic leaves of $M_{\ge0}(\Sigma)$ are naturally labelled by compatible subsets of cluster variables, where the number of cluster variables in the labeling set determines the corank of the symplectic leaf.
Here there is a unique dense symplectic leaf and the boundary of $M_{\ge0}(\Sigma)$ is again a union of symplectic leaves of lower dimension where the Poisson structure degenerates.

\begin{theorem}
  Let $\Sigma$ be a seed.  
  The 1-skeleton of $M_{\ge0}(\Sigma)$ given by 0-dimensional and 1-dimensional symplectic leaves identifies with the exchange graph of $\cA(\Sigma)$.  
  Moreover, if $\Sigma$ is a seed of finite-type, then $M_{\ge0}(\Sigma)$ provides a realization of the generalized associahedron with the same Cartan type as $\Sigma$.
\end{theorem}
\begin{proof}
  The 0-dimensional symplectic leaves correspond to the vanishing of all cluster variables from a seed mutation equivalent to $\Sigma$.  
  Then a 1-dimensional symplectic leaf whose boundaries correspond to seeds $\Sigma'$ and $\Sigma''$ exactly corresponds to the non-vanishing of exchangable cluster variables $x'_k$ and $x''_k$.
  But this is exactly the exchange graph of $\cA(\Sigma)$.

  When $\Sigma$ is of finite-type, the realization of $M_{\ge0}(\Sigma)$ as a simplicial complex, given by taking symplectic leaves as cells, is naturally dual to the cluster complex of $\cA(\Sigma)$, i.e. $M_{\ge0}(\Sigma)$ identifies with the associated generalized associahedron.
\end{proof}

%%%%%%%%%%%%%%%%%%%%%%%%%%%%%%%%%%%%%%%%%%%%%%%%%%%%%%%%%%%%%%%%
\section{Symplectic Topology of the Nonnegative Cluster Groupoid}
Let $\cG_{\ge0}(\Sigma)$ denote the symplectic groupoid over $M_{\ge0}(\Sigma)$.  
In this section we introduce a Poisson spray which may be used to construct $\cG_{\ge0}(\Sigma)$ and apply a Moser argument to show that up to symplectomorphism $\cG_{\ge0}(\Sigma)$ can be identified with $T^*M_{\ge0}(\Sigma)$.

\bibliographystyle{hyperamsplain}
\bibliography{cluster_symplectic}

\end{document}
